\section{The Derived Vector of a Parametrized Curve.} Consider a function whose domain is a subset of the set of all real numbers and whose range is a subset of all vectors in the plane. If we denote this function by $\mbox{\bf{v}}$, then its value at each number $t$ in the domain is the vector $\mbox{\bf{v}}(t)$. Every such vector-valued function $\mbox{\bf{v}}$ of a real variable defines two real-valued \textbf{coordinate functions} $v_1$ and $v_2$ as follows: For every $t$ in the domain of $\mbox{\bf{v}}$, the numbers $v_1(t)$ and $v_2(t)$ are the first and second coordinates of the vector $\mbox{\bf{v}}(t)$, respectively. Hence, if the initial point of $\mbox{\bf{v}}(t)$ is $P(t)$, then $v_1(t)$ and $v_2(t)$ are defined by the equation

\begin{equation}
\mbox{\bf{v}}(t) = (v_1(t), v_2(t))_{P(t)}. 
\label{eq10.4.1}
\end{equation}


Limits of vector-valued functions are defined in terms of limits of real-valued functions. Specifically, the \textbf{limit} of $\mbox{\bf{v}}(t)$, as $t$ approaches $t_0$, will be denoted by $\lim_{t \rightarrow t_0} \mbox{\bf{v}}(t)$ and is defined by

\begin{equation}
\lim_{t \rightarrow t_0} \mbox{\bf{v}}(t) = (\lim_{t \rightarrow t_0} v_1(t), \lim_{t \rightarrow t_0} v_2(t)) _{\lim_{t \rightarrow t_0} P(t)} .  
\label{eq10.4.2}
\end{equation}
[For the definition of $\lim_{t \rightarrow t_0} P(t)$, see page 542.] There is the possibility that all the vectors $\mbox{\bf{v}}(t)$ have the same initial point $P_0$, i.e., that they all lie in the vector space $\mathcal{V}_{P_0}$. If this happens, (2) reduces to the simpler equation 
$$
\lim_{t \rightarrow t_0} \mbox{\bf{v}}(t) = (\lim_{t \rightarrow t_0} v_1(t), \lim_{t \rightarrow t_0} v_2(t))_{P_0} .
$$

Let $C$ be a curve in the plane defined by a parametrization $P: I \rightarrow R^2$. If the coordinate functions of $P$ are denoted by $x$ and $y$, then $C$ is the set of all points

$$
P(t)= (x(t), y(t)) 
$$
such that $t$ is in the interval $I$. A typical example is shown in Figure 14. Consider a number $t_0$ in $I$. If $t$ is in $I$ and distinct from $t_0$, then the vector $(P(t_0), P(t))$ represents the change in the value of $P$ from the point $P(t_0)$ to
%570 GEOMETRY IN THE PLANE [CHAP. 1O
the point $P(t)$. Thus for a change in the value of the parameter from $t_0$ to $t$, the scalar product

\begin{equation}
\frac{1}{t - t_0} (P(t_0), P(t))  
\label{eq10.4.3}
\end{equation}


%Figure 14
\putfig{2.0truein}{scanfig10_14}{}{fig 10.14}

\noindent is the ratio of the corresponding change in the value of $P$ to the difference $t - t_0$. Hence the vector (3) represents an average rate of change in position with respect to a change in the parameter. In analogy with the definition of the derivative of a real-valued function, we define the \textbf{derived vector} of $P$ at $t_0$, denoted by $\mbox{\bf{d}}P(t_0)$, by the equation


$$
\mbox{\bf{d}}P(t_0)= \lim_{t \rightarrow t_0} \frac{1}{t - t_0} (P(t_0),P(t)). 
$$

Since $P(t_0) = (x(t_0), y(t_0))$ and $P(t) = (x(t), y(t))$, the coordinate form of the vector $(P(t_0), P(t))$ is given by

$$
(P(t_0), P(t)) = (x(t) - x(t_0), y(t) - y(t_0))_{P(t_0)}.
$$
By the definition of the scalar product,

$$
\frac{1}{t - t_0} (P(t_0), P(t)) = \left( \frac{x(t) - x(t_0)}{t - t_0}, \frac{y(t) - y(t_0)}{t - t_0} \right)_{P(t_0)} ,
$$
and so
$$
\mbox{\bf{d}}P(t_0) = \left( \lim_{t \rightarrow t_0} \frac{x(t) - x(t_0)}{t - t_0}, \lim_{t \rightarrow t_0} \frac{y(t) - y(t_0)}{t - t_0} \right)_{P(t_0)} .
$$
Recall that the derivatives of the functions $x$ and $y$ at $t_0$ are by definition
\begin{eqnarray*}
x'(t_0) &=& \lim_{t \rightarrow t_0} \frac{x(t) - x(t_0)}{t - t_0}, \\
y'(t_0) &=& \lim_{t \rightarrow t_0} \frac{y(t) - y(t_0)}{t - t_0}, 
\end{eqnarray*}
%sec. 4] THE DeRIveD VECTOR OF A PARAMETRIZED CURVE  571
provided these limits exist. It follows that

\begin{theorem} The parametrization defined by $P(t) = (x(t), y(t) )$ is differentiable at $t_0$ if and only if the derived vector $\mbox{\bf{d}}P(t_0)$ exists. If it does exist, then
$$
\mbox{\bf{d}}P(t_0) = (x'(t_0), y'(t_0))_{P(t_0)} .
$$
\end{theorem}

%EXAMPLE 1. 
\begin{example} Consider the curve parametrized by
$$
P(t) = (x(t), y(t)) = (t^2 - 1, 2t + 1), \;\;\; - \infty < t < \infty.
$$
Compute the derived vectors of $P$ at $t_0 = - 1$, at $t_0 = 0$, and at $t_0 = 1$. Draw the curve and the three derived vectors in the $xy$-plane. As a result of (4.1), we have
$$
\mbox{\bf{d}}P(t_0) = (x'(t_0),y'(t_0))_{P(t_0)} = (2t_0, 2)_{P(t_0)} .
$$
Hence 
$$
\begin{array}{ccl}
\mbox{\bf{d}}P(-1) = (- 2, 2)_{P(-1)}&\;\;\;\mathrm{and}&\;\;\; P(-1) = (0, -1), \vspace{.1in}\\
\mbox{\bf{d}}P(0) = (0, 2)_{P(0)}     &\;\;\;\mathrm{and}&\;\;\; P(0) = (-1, 1), \vspace{.1in}\\
\mbox{\bf{d}}P(1) = (2, 2)_{P(1)}     &\;\;\;\mathrm{and}&\;\;\; P(1) = (0, 3) .
\end{array}
$$
The terminal points of the three derived vectors are, respectively,
\begin{eqnarray*}
(0 - 2, -1 + 2) &=& (-2, 1), \vspace{.1in}\\
(-1 + 0, 1 + 2) &=& (-1, 3), \vspace{.1in}\\
(0 + 2, 3 + 2) &=& (2, 5) . 
\end{eqnarray*}
The parametrized curve is a parabola, as can be seen by setting
$$
\left \{ \begin{array}{l}
x = t^2 - 1, \\
y = 2t + 1.
\end{array}
\right .
$$
Solving the second equation for $t$, we get $t = \frac{y -1}{2}$, and substituting this value in the first, we obtain $x = \frac{(y - 1)^2}{4} - 1$, or, equivalently,
$$
4(x + 1) = (y - 1)^2.
$$
The latter is an equation of a parabola with vertex (-1, 1). If $x = 0$, then $4 = (y -  1)^2$, or, equivalently, $\pm 2 = y -1$, which implies that $y = -1$ or 3. The parametrized curve together with the three vectors is shown in Figure 15. Note that each of these vectors is tangent to the parabola.
\end{example}

If a parametrization $P: I \rightarrow R^2$ is differentiable at $t_0$, then we define a \textbf{tangent vector} to the resulting parametrized curve at $t_0$ to be any scalar multiple of the derived vector $\mbox{\bf{d}}P(t_0)$. In particular, the derived vector itself is a tangent vector. The set of all tangent vectors at $t_0$ is a subset of $\mathcal{V}_{P(t_0)}$, since every scalar multiple of $\mbox{\bf{d}}P(t_0)$ has initial point $P(t_0)$. Moreover,

%572 GEOMETRY IN THE PLANE [CHAP. 1O

%Figure 15
\putfig{3.75truein}{scanfig10_15}{}{fig 10.15}

\begin{theorem} The set of all tangent vectors to the parametrized curve $P(t)$ at $t_0$ is a vector space.
\end{theorem}

\begin{proof}
This result has nothing to do with any special properties of the derived vector, since the set of all scalar multiples of \textit{any} vector $\mbox{\bf{u}}$ is a vector space. This result is proved, if $\mbox{\bf{u}}$ is nonzero, in Example 3 of Section 3. If $\mbox{\bf{u}}$ is a zero vector, the result is even simpler: The set of all scalar multiples of a zero vector $\mbox{\bf{0}}$ is the set having $\mbox{\bf{0}}$ as its only member, and the six conditions for a vector space are trivially satisfied. This completes the argument.
\end{proof}

Consider a parametrization defined by $P(t)= (x(f), y(t))$, which is differentiable at $t_0$ and for which the derived vector $\mbox{\bf{d}}P(t_0)$ is nonzero. If we set $x'(t_0) = d_1$ and $y'(t_0) = d_2$, then 
$$
\mbox{\bf{d}}P(t_0) = (d_1, d_2)_{P(t_0)},
$$
where not both coordinates $d_1$ and $d_2$ are zero. The set of all tangent vectors at to is the set of all scalar multiples 
$$
s\mbox{\bf{d}}P(t_0) = (sd_1, sd_2)_{P(t_0)},
$$
%SEC. 4] THE DERIVED VECTOR OF A PARAMETRIZED CURVE  573
where $s$ is any real number. If $P(t_0) = (a, b)$, then the terminal point of $s\mbox{\bf{d}}P(t_0)$ is equal to
$$
(sd_1 + a, sd_2 + b).
$$
Hence the set of all terminal points of tangent vectors at $t_0$ is the set of all points $(x, y)$ such that 

\begin{equation}
\left \{ \begin{array}{l}
x = sd_1 + a,\\
y = sd_2 + b,
\end{array}
\right .
\label{eq10.4.4}
\end{equation}
where $s$ is any real number and $d_1$ and $d_2$ are not both zero. It is easy to verify that this set is a straight line (see Problem 4). We conclude that \textit{if the derived vector $\mbox{\bf{d}}P(t_0)$ exists and is nonzero, then the set of all terminal points of the tangent vectors at $to$ to the curve parametrized by $P$ is a straight line.}  It is called the \textbf{tangent line} to the parametrized curve at $t_0$.

%EXAMPLE 2. 
\begin{example} Consider the ellipse defined parametrically by 
$$
P(t) = (x(t), y(t)) = (4 \cos t, 2 \sin t),
$$
for every real number $t$. Compute the derived vector at $t_0 = \frac{\pi}{6}$, and draw it and the ellipse in the $xy$-plane. In addition, write an equation for the tangent line at $t_0 = \frac{\pi}{6}$, and draw the tangent line in the figure. The derived vector is easily computed: 
\begin{eqnarray*}
\mbox{\bf{d}}P(t_0) &=& (x'(t_0), y'(t_0))_{P(t_0)} = (- 4 \sin t_0, 2 \cos t_0)_{P(t_0)} \\
&=& \Big(- 4\sin\frac{\pi}{6}, \cos\frac{\pi}{6} \Big) = (-2, \sqrt{3})_{P(t_0)}, 
\end{eqnarray*}

\noindent where
$$
P(t_0) = \Big(4\cos\frac{\pi}{6}, 2 \sin\frac{\pi}{6} \Big) = (2\sqrt{3}, 1).
$$
The terminal point of the derived vector is therefore equal to 
$$
(2\sqrt{3} - 2, 1 + \sqrt 3).
$$

The parametrization $P$ can also be written in terms of the equations 
$$
\left \{ \begin{array}{ll}
x = 4 \cos t,        & \\
y = 2 \sin t, \;\;\; &-\infty < t < \infty, 
\end{array}
\right .
$$
from which it follows that
$$
\frac{x^2}{4^2} + \frac{y^2}{2^2} = \cos^{2} t + \sin^{2} t = 1.
$$
%574 GEOMETRY IN THE PLANE [CHAP. 10
Hence every point $(x, y)$ on the parametrized curve satisfies the equation

\begin{equation}
\frac{x^2}{4^2} + \frac{y^2}{2^2} = 1 .
\label{eq10.4.5}
\end{equation}
Conversely, it can be shown (as in Example 2, page 544) that any ordered pair $(x, y)$ which satisfies (5) also lies on the parametrized curve. We recognize (5) as an equation of the ellipse shown in Figure 16. The derived vector $\mbox{\bf{d}}P(t_0)$ and the tangent line at $\frac{\pi}{6}$ are also shown in the figure.

%Figure 16
\putfig{4.5truein}{scanfig10_16}{}{fig 10.16}

If $s$ is an arbitrary real number, then the scalar product $s\mbox{\bf{d}}P(t_0)$ in this example is the vector
$$
s\mbox{\bf{d}}P(t_0) = s(- 2, \sqrt 3)_{P(t_0)} = (- 2s, \sqrt{3} s)_{P(t_0)}.
$$
The terminal point of this vector, since $P(t_0) = (2\sqrt{3}, 1)$, is the point
$$
(-2s + 2\sqrt 3, \sqrt{3} s + 1).
$$
Hence the tangent line at $\frac{\pi}{6}$ is parametrically defined by the equations

\begin{equation}
\left \{ \begin{array}{l}
x = - 2s + 2 \sqrt 3 ,\\
y= \sqrt {3} s + 1, \;\;\; -\infty < s < \infty .
\end{array}
\right .
\label{eq10.4.6}
\end{equation}
%SEC. 4] THE DERIVED VECTOR OF A PARAMETRIZED CURVE  575
Solving the first of these for $s$, we obtain $s = \frac{-x + 2\sqrt 3}{2}$. Substitution in the second then yields


%(7)
\begin{eqnarray*}
y &=& \sqrt 3 \Big(\frac{-x + 2\sqrt 3}{2} \Big) + 1,\\
y &=& - \frac{\sqrt 3}{2}x + 4.  \mbox{\hspace{2in} (7)}
\end{eqnarray*}
Thus any point on the tangent line satisfies (7), and it is easy to verify that, for any $x$ and $y$ which satisfy (7), there is a unique s such that equations (6) hold. We conclude that (7) is an equation of the tangent line.
\end{example}

It is important to know that the ideas introduced in this section are consistent with related concepts developed earlier. For example, consider a differentiable parametrization defined by
$$
P(t) = (x(t), y(t)), \;\;\;\mathrm{for every}~t~\mathrm{in some interval}~I. 
$$
Suppose that, for some $t_0$ in $I$, there exists a differentiable function $f$ such that 
$$
y(t) = f(x(t)),
$$
for every $t$ in some subinterval of $I$ containing $t_0$ in its interior. This situation was described in Section I and was illustrated in Figure 3 (page 545). If such a function $f$ exists, we say that $y$ is a differentiable function of $x$ on the parametrized curve $P(t)$ in a neighborhood of $P(t_0)$. Formulas (5) and (6), page 546, assert that, for every $t$ in the subinterval, 
$$
\frac{dy}{dx} = f'(x(t)) = \frac{y'(t)}{x'(t)},
$$

\noindent provided $x'(t) \neq 0$. Hence $\frac{y'(t)}{x'(t)}$ is the slope of the line tangent to the graph of $f$ at the point   
$$
(x(t), f(x(t))) = (x(t), y(t)) = P(t) .
$$
Moreover, in the vicinity of $P(t_0)$, the graph of $f$ is the curve parametrized by $P$. At every $t$ in the subinterval, the derived vector of $P$ is equal to
$$
\mbox{\bf{d}}P(t) = (x'(t), y (t))_{P(t)} .
$$
This vector is, by definition, a tangent vector to the parametrized curve. Its initial point is $P(t) = (x(t), y(t))$ and its terminal point is
$$
Q(t) = (x(t) + x'(t), y(t) + y'(t)).
$$
The slope of the line segment joining these two points is given by 
$$
m(P(t), Q(t)) = \frac{(y(t) + y'(t)) - y(t)}{(x(t) + x'(t)) - x(t)} = \frac{y'(t)}{x'(t)},
$$
%576 GEOMETRY IN THE PLANE [CHAP. 1O 
provided $x'(t) \neq 0$. We conclude that the concept of tangency, as defined in terms of the derived vector to a parametrized curve, is consistent with the earlier notion, defined in terms of the derivative.
 
