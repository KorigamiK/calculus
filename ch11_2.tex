\section{First-Order Linear Differential Equations.} A differential equation which can be written in the form
$$
a_1(x) \frac{dy}{dx} + a_0(x)y = b(x),
$$
where $a_1, a_0$, and $b$ are given functions of $x$ and where $a_1$ is not the zero function, is a \textbf{first-order linear ditTerential equation.} In this section we shall show
%624 DIFFERENTIAL EQUATIONS [CHAP. 11
how to obtain the general solution of equations of this type. Since $a_1$ is by assumption not the zero function, we can divide both sides of the above equation by $a_1(x)$. Setting $\frac{a_0(x)}{a_1(x)} = P(x)$ and $\frac{b(x)}{a_1(x)} = Q(x)$, we therefore obtain the differential equation
\begin{equation}
\frac{dy}{dx} + P(x)y = Q(x),   
\label{eq11.2.1}
\end{equation}
which is the form we shall use in deriving the solution. We shall assume that the functions $P$ and $Q$ are continuous, thus assuring ourselves that they have antiderivatives.

Let us suppose that the function $y = f(x)$ is a solution to the differential equation (1). We shall derive a formula which expresses $y$ in terms of $P$ and $Q$ and a constant $c$ of integration. Conversely, it will be a simple matter to verify that any function $y$ defined by this formula is a solution to (1). Hence the formula gives the general solution to the differential equation.

The derivative of the product of $y$ and a function $\varphi$ is given by
\begin{equation}
\frac{d}{dx} (\varphi(x)y) = \varphi(x) + \varphi'(x)y. 
\label{eq11.2.2}
\end{equation}
Note that the first term on the right has $\frac{dy}{dx}$ as a factor and the second has $y$ as a factor, and that the same is true of the two terms on the left side of equation (1). This fact suggests seeking a function $\varphi$ which has the property that, if both sides of (1) are multiplied by $\varphi(x)$, then the left side of the resulting equation is the derivative of the product $\varphi(x)y$. If both sides of (1) are multiplied by an arbitrary $\varphi(x)$, the result is
\begin{equation}
\varphi(x) \frac{dy}{dx} + \varphi(x)P(x)y = \varphi(x)Q(x) . 
\label{eq11.2.3}
\end{equation}
Comparison of this equation with (2) shows that its left side is equal to $\frac{d}{dx} (\varphi(x)y)$ provided $\varphi(x)P(x)y = \varphi'(x)y$, which will in turn be true provided
\begin{equation}
\varphi(x)P(x) = \varphi'(x).  
\label{eq11.2.4}
\end{equation}

However, it is easy to find a function $\varphi$ which satisfies (4), since, as a differential equation with $\varphi$ the unknown function, it is separable. Solving it, we obtain
$$
\frac{\varphi'(x)}{\varphi(x)} = P(x)
$$
%SEC. 21 F I PST-ORDER L INEA R DI F FER ENTIAL EQU ATIONS 625
Whence 
$$
\int \frac{\varphi'(x)}{\varphi(x)} dx = \int P(x) dx ,
$$
which implies 
$$
\ln |\varphi(x)| = \int P(x) dx + c, 
$$
and so 
$$
|\varphi(x)| = e^{\int P(x)dx+c} .
$$
Since we are only seeking a solution to (4), and not the most general form of the solution, we may assume that $\varphi(x)$ is positive and also ignore the constant of integration. We conclude that if
\begin{equation}
\varphi(x) = e^{\int P(x) dx},
\label{eq11.2.5}
\end{equation}
\textit{then the left side of eguation (3) is equal to} $\frac{d}{dx} (\varphi(x)y)$.

With (5), equation (3) therefore becomes
$$
\frac{d}{dx} (\varphi(x)y) = \varphi(x)Q(x). 
$$
integration yields 
$$
\varphi(x)y = \int \varphi(x) Q(x) dx + c,
$$
and so 
$$
y = \frac{1}{\varphi(x)} \Big[ \int \varphi(x) Q(x) dx + c \Big] ,
$$
for some real number $c$. Replacing $\varphi(x)$ by $e^{\int P(x)dx}$, we obtain the promised formula:
\begin{theorem}
$$
y = e^{- \int P(x)dx} \Big[\int e^{\int P(x)dx} Q(x) dx + c \Big]  .
$$
\end{theorem}

Suppose next that $c$ is an arbitrary constant and that the function $y$ is
dcfined by (2.1). Then
$$
ye^{\int P(x)dx} = \int e^{\int P(x)dx} Q(x) + c.
$$
%626 DIFFERENTIAL EQUATIONS [ClIAP. 1 1
The derivative of the right side of this equation is $e^{\int P(x)dx}Q(x)$ and that of the left side is
$$
\frac{dy}{dx} e^{\int P(x)dx} + yP(x)e^{\int P(x)dx} .
$$
Hence
$$
e^{\int P(x)dx} \Big[\frac{dy}{dx} + P(x)y \Big] = e^{\int P(x)dx} Q(x), 
$$
which implies at once that $\frac{dy}{dx} + P(x)y = Q(x)$; i.e., $y$ is a solution to (1). We conclude that formula (2.1) gives the general solution to the differential equation (1).

We strongly recommend that no one memorize (2.1). The important fact to remember is that, if the first-order linear differential equation $\frac{dy}{dx} + P(x)y = Q(x)$ is multiplied through by $\varphi(x) = e^{\int P(x)dx}$, then the left side of the resulting equation is equal to the derivative of the product $\varphi(x)y$. Consequently, the new equation can be integrated to give
$$
\varphi(x)y = \int \varphi(x) Q(x) dx + c,
$$
which can then be solved for $y$. This function $\varphi(x) = e^{\int P(x)dx}$, which enables us to change a seemingly nonintegrable sum into the derivative of a product by multiplication, is called an \textbf{integrating factor.}


\begin{example} %EXAMPLE 1. 
Solve the differential equation 
$$
x^2 \frac{dy}{dx} - 3xy - 2x^2 = 4x^4 .
$$
To put this in the form of (1), we add $2x^2$ to both sides and then divide by $x^2$. The result is
\begin{equation}
\frac{dy}{dx} - \frac{3}{x} = 4x^2 + 2,  
\label{eq11.2.6}
\end{equation}
where $P(x) = -\frac{3}{x}$ and $Q(x) = 4x^2 + 2$. An antiderivative of $P(x)$ is given by  
$$
\int P(x) dx = \int -\frac{3}{x} dx = -3 \ln |x| = \ln |x^{-3}|,
$$
and it follows that the function
$$
\varphi(x) = e^{\int P(x)dx} = e^{\ln |x^{-3}|} = |x^{-3}|
$$
% SEC. 2] FIRST-ORDER LTNEAR DIFFERENTIAL EQUATIONS  627
is an integrating factor. Equation (4) shows that if $\varphi(x)$ is an integrating factor, then so also is $-\varphi(x)$. Hence we may drop the absolute values and write simply
$$
\varphi(x) = x^{-3}. 
$$
Multiplying both sides of (6) by $x^{-3}$, we obtain 
$$
x^{-3} \frac{dy}{dx} - 3x^{-4}y = 4x^{-1} + 2x^{-3} ,
$$
It is easy to see that the left side of this equation is equal to $\frac{d}{dx}(x^{-3}y)$. Hence
$$
\frac{d}{dx} (x^{-3} y) = 4x^{-1} + 2x^{-3}, 
$$
and so 
\begin{eqnarray*}
x^{-3}y &=& \int (4x^{-1} + 2x^{-3}) dx + c\\
            &=& 4 \ln |x| + 2 \frac{x^{-2}}{-2} + c \\
            &=& 4 \ln |x| - \frac{1}{x^2} + c,
\end{eqnarray*}
where $c$ is an arbitrary constant. It follows that 

$$
y = 4x^{3} \ln |x| - x + cx^{3}
$$ 
is the general solution.
\end{example}

%EXAMPLE 2. 
\begin{example} Find the general solution of the differential equation 
$$
\frac{dy}{dx} + 3y = 2 \sin x.
$$
Note that this is a first-order linear differential equation with constant coefficients, but that it is not homogeneous, because the right side is not the zero function. In this example we have $P(x) = 3$ and $Q(x) = 2 \sin x$. Hence
$$
\int P(x)dx = \int 3 dx = 3x ,
$$ 
and an integrating factor is
$$
\varphi(x) =  e^{\int P(x)dx} = e^{3x} . 
$$
lt follows that
$$
\frac{d}{dx} (e^{3x} y) = 2e^{3x} \sin x,
$$
% 628 DIFFERENTIAL EQ UA TIONS [CHAP. 
and so 
\begin{equation}
e^{3x} y = \int 2e^{3x} \sin x dx + c.  
\label{eq11.2.7}
\end{equation}
To evaluate $\int 2e^{3x} \sin xdx = 2\int e^{3x} \sin x dx$, we use integration by parts twice:
\begin{eqnarray*}
\int e^{3x} \sin x dx &=& - \int e^{3x} d \cos x \\
&=& -e^{3x} - \cos x + \int \cos x de^{3x} \\
&=& -e^{3x} \cos x + 3 \int e^{3x} \cos x dx. \\
\int e^{3x} \cos x dx &=& \int e^{3x} d \sin x \\
&=& e^{3x} \sin x - \int \sin x de^{3x}\\
&=& e^{3x} \sin x - 3\int e^{3x} \sin x dx.
\end{eqnarray*}
Combining these results, we get
$$
\int e^{3x} \sin x dx = - e^{3x} \cos x  + 3e^{3x} \sin x - 9 \int e^{3x} \sin x dx,
$$
whence
$$
10 \int e^{3x} \sin x dx = e^{3x}(3 \sin x - \cos x),
$$
and so
$$
2\int e^{3x} \sin x dx = \frac{e^{3x}}{5} (3 \sin x - \cos x).
$$
Returning to (7), we have
$$
e^{3x}y = \frac{e^{3x}}{5} (3 \sin x - \cos x) + c,
$$
and consequently the general solution of the differential equation is given by 
\begin{equation}
y = \frac{1}{5} (3 \sin x - \cos x) + ce^{-3x}  ,  
\label{eq11.2.8}
\end{equation}
where $c$ is an arbitrary constant.
\end{example}

Note that the above solution (8) of the differential equation of Example 2 is the sum of two terms. The second, which is $ce^{-3x}$, is thc general solution
%SEC 2] FIRST-ORDER LINEAR DIFFERENTIAL EQUATIONS  629
of the homogeneous differential equation $\frac{dy}{dx} + 3y = 0$. The first term, $\frac{1}{5}(3 \sin x - \cos x)$, is one particular solution of the nonhomogeneous differential equation $\frac{dy}{dx} + 3y = 2 \sin x$. As we shall see, this situation is typical of the solutions of linear differential equations.

