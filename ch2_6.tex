\section{The Differential.}\label{sec 2.6}
If $y = f(x)$, we have denoted the derivative of $f$ by $f'$, or $\frac{df}{dx}$, or $\frac{dy}{dx}$. The value of the derivative at a number a is written $f'(a)$, or $\frac{df}{dx} (a)$, or $\frac{dy}{dx} (a)$. Up to this point, the expressions $df$, $dy$, and $dx$ by themselves have had no meaning other than as parts of notations for the derivative. However, the cancellation suggested by the Chain Rule
$$
\frac{dy}{dx} = \frac{dy}{du} \frac{du}{dx}
$$
\noindent indicates that the derivative behaves like a ratio and suggests that it may be possible to sensibly regard it as such. In this section we shall define a mathematical object called the differential of a function, examples of which are $df$, $dy$, and $dx$. The ratio of $df$, or $dy$, to $dx$ will be equal to the derivative.

\putfig{3.75truein}{scanfig2_20}{}{fig 2.20}

If $f$ is a function having a derivative at $a$, we define its \textbf{differential at} $a$, denoted by $d_{a}f$,  to be the linear function whose value for any number $t$ is
$$
(d_{a}f)(t) = f'(a)t.
$$
For example, if $f(x) = x^2 - 2x$, then the differential $d_{a}f$ is the function of $t$ defined by $f'(a)t =  (2a - 2)t$. In particular,

$$
(d_{3}f)(t)= [2 \cdot 3 - 2]t = 4t.
$$
\noindent The value of the differential for a typical function $f$ is illustrated in Figure \f{2.20}.

By simply $df$ we mean the rule (or function) that assigns the linear function $d_{x}f$ to each number $x$ in the domain of $f'$.

\begin{theorem}
If $f$ and $u$ are differentiablefunctions, then 

\begin{equation}
 df(u) = f'(u)du. 
\label{eq2.6.1}
\end{equation}
\end{theorem}
\begin{proof}
This formula is an abbreviation of the equation 
\begin{equation}
d_{x} f(u) = f'(u(x)) d_{x}u.  
\label{eq2.6.2}
\end{equation}
The proof is an application of the Chain Rule.
We first write down the two linear functions $d_{x}f(u)$ and $d_{x}u$. By the definition of the differential they are
\begin{eqnarray*}
   (d_{x}u)(t) &=& u'(x)t,\\
(d_{x}f(u))(t) &=& [[f(u)]'(x)]t.
\end{eqnarray*}
The Chain Rule says that $[f(u)]'(x) = f'(u(x))u'(x)$. Hence
\begin{eqnarray*}
(d_{x} f(u) )(t) &=& [f'(u(x) )u'(x)]t  \\
                        &=& f'(u(x))(d_{x}u)(t).
\end{eqnarray*}
Thus (2) appears as an equality between linear functions,
and the proof is complete.
\end{proof}

If $u$ is the independent variable $x$, then (6.1) reduces to the formula

\begin{equation}
 df(x) = f'(x)dx. 
\label{eq2.6.3}
\end{equation}

\begin{example}
%EXAMPLE 1. 
Evaluate the following differentials:

\begin{quote}
\begin{description}
\item[(a) $d(x^2 + 2)$,]
\item[(b) $d\sqrt{x^2 + 3}$, ]
\item[(c) $d(2x^2-x)^7$.] 
\end{description}
\end{quote}

\noindent Using formula (3), we get immediately 

\begin{quote}
\begin{description}
\item[(a') $d(x^2 + 2) = 2xdx$]
\item[(b') $d\sqrt{x^2 + 3} = x(x^2 + 3)^{-1/2} dx$]
\item[(c') $d(2x^2-x)^7 = 7(2x^2-x)^6(4x-1)dx$.]
\end{description}
\end{quote}

\noindent It is worthwhile learning to use the stronger formula (1). In problem (b), let $f$ be the function $f(u) = \sqrt{u}$. If we set $u = x^2 + 3$, then $du = 2x dx$ and

\begin{eqnarray*}
d\sqrt{x^2 + 3} &=& df(u) = f'(u) du\\
                        &=& \frac{1}{2}u^{-1/2}du\\
                        &=& \frac{1}{2}(x^2 + 3)^{-1/2}2x dx.
\end{eqnarray*}
\noindent Let us also do problem (c) using (1), but without explicitly making the substitution $u = 2x^2 - x$. We get

\begin{eqnarray*}
d(2x^2 - x)^7 &=& 7(2x^2 - x)^{6} d(2x^2 - x)\\
                     &=& 7(2x^2 - x)^{6} (4x - 1) dx.
\end{eqnarray*}
%118 APPLICATIONS OF THE DERIVATIVE [CHAP. 2

Formula (3) establishes the fact that the ratio of $df$ to $dx$ is equal to the derivative $f'$. We can see this in greater detail by going back to the definitions:

\begin{eqnarray*}
(d_{a}f)(t) &=& f'(a)t, \\
(d_{a}x)(t) &=& x'(a)t.
\end{eqnarray*}

\noindent Since $x$ is the identity function, its derivative is the constant function 1. Hence $(d_{a}x)(t) = t.$ The ratio of the two linear functions $d_{a}f$ and $d_{a}x$ is thus the constant function
$$
\frac{d_{a}f}{d_{a}x} = \frac{f'(a)t}{t} = f'(a).
$$
Having proved this formula for every a in the domain of $f'$, we can write it simply as
$$
\frac{df}{dx} (a) = f'(a) \;\;\; \mbox{or} \;\;\; \frac{df}{dx} = f'.
$$

If $f$ and $g$ are differentiable functions, then it is easy to show that $d_{a}(f + g) = d_{a}f + d_{a}g$. The proof involves only the definition of the differential plus the fact that the derivative of a sum is the sum of the derivatives. In detail:

\begin{eqnarray*}
[d_{a}(f + g)](t) &=& [(f + g)'(a)] t = [f' (a) + g'(a)] t\\
                         &=& f'(a)t + g'(a)t = (d_{a}f)(t) + (d_{a}g)(t)\\
                         &=& [d_{a}f + d_{a}g](t) .
\end{eqnarray*}
The result is simply the equation
$$
d(f + g) = df + dg.
$$
An analogous argument using the Product Rule for differentiation shows that $d_{a}(fg) = f(a)d_{a}g + g(a)d_{a}f$, or, more simply,
$$
d(fg) = f dg + g df.
$$
For each one of the six differentiation rules \thref{ddxrules}
proved in Section \secref{1.7} of Chapter \chref{1},
there is an analogous rule in terms of differentials:
Let $u$ and $v$ be differentiable functions, and $c$ a constant. Then
\begin{prop}
\label{thm drules}
\begin{quote}
\begin{description}
\item [(i)] $d(u + v) = du + dv$,
\item [(ii)] $d(cu)= cdu$,
\item [(iii)] $dc= 0$,
\item [(iv)] $d(uv) = u dv + v du$,
\item [(v)]
$du^r = ru^{r-1}du, \provx{where $r$ is any rational number,}$  
\item [(vi)] $d \biggl(\frac{u}{v} \biggr) = \frac{vdu - udv}{v^2}$.
\end{description}
\end{quote} 
\end{prop}

Note that we have replaced the analogue of (v)
in the list in Section \secref{1.7} of Chapter \chref{1}
by the formula corresponding to the more powerful theorem
\thref{1.8.2} of Chapter \chref{1}.
\end{example}

\begin{example}
%EXAMPLE 2. 
Find the differential $d(x^3 + \sqrt{x^2 + 2x})^7$. Applying the above formulas successively, we get

\begin{eqnarray*}
d(x^3 + \sqrt{x^2 + 2x})^7
&=& 7(x^3 + \sqrt{x^2 + 2x})^6 d(x^3 + \sqrt{x^2 + 2x})  \;\;\;\mbox{by (v)}\\
&=& 7(x^3 + \sqrt{x^2 + 2x})^6 (dx^3 + d\sqrt{x^2 + 2x}) \;\;\;\mbox{by (i)}\\
&=& 7(x^3 + \sqrt{x^2 + 2x})^6 [3x^2 dx + \frac{1}{2}(x^2 + 2x)^{-1/2} d(x^2 + 2x)] \\
& & \hspace{2.5in} \mbox{by (v)}\\
&=& 7(x^3 + \sqrt{x^2 + 2x})^6 [3x^2 dx + \frac{1}{2}(x^2 + 2x)^{-1/2} (2x dx + 2 dx)] \\ 
& &\hspace{2in} \mbox{by (i) and (ii)} \\ 
&=& 7(x^3 + \sqrt{x^2 + 2x})^6 \biggl( 3x^2 +\frac{x+1}{\sqrt{x^2 +
2x}}\biggr) dx.
\end{eqnarray*}
\noindent The derivative is therefore given by
$$
\frac{d(x^3 + \sqrt{x^2 + 2x})^7}{dx} = 7(x^3 + \sqrt{x^2 + 2x})^6 \biggl(3x^2 + \frac{x +
1}{\sqrt{x^2 + 2x}}\biggr).
$$
\end{example}

The task of computing the differential of a complicated function of $x$ amounts to successively working the differential operator d through the given expression from left to right. At each stage one uses the correct one of formulas (i) through (vi), or formula (1), until one finally reaches $dx$, and the process stops. The derivative can then be obtained by dividing the resulting equation by $dx$. Note that an equation of the form $df(x) = ...$ will always contain the symbol $d$ on the right side. Equations such as $dx^5 = 5x^4$ are not only false; they are nonsense. (Correct version: $dx^5 = 5x^4 dx$.)

%EXAMPLE 3. 
\begin{example} 
Consider the functions 

\begin{quote}
\begin{description}
\item [(a) $y = (4x^3 + 3x^2 + 1)^2,$]
\item [(b) $y = \frac{x^2 -1}{x^2 + 1}$]
\item [(c) $z = 3y^{5/3}.$]
\end{description}
\end{quote}
\noindent Find the differential of each:

\begin{eqnarray*}
\mbox{(a')}\; dy &=& 2(4x^3 + 3x^2 + 1) d(4x^3 + 3x^2 + 1) \\
           &=& 2(4x^3 + 3x^2 + 1)(12x^2 dx + 6x dx) \\
           &=& 12x(4x^3 + 3x^2 + 1)(2x + 1) dx,
\end{eqnarray*}
%120 APPLICA TIONS OF THE DERI VATI VE [CHAP. 2
\begin{eqnarray*}
\mbox{(b')}\; dy &=& \frac{(x^2 + 1 ) d(x^2 - 1 ) - (x^2 - 1 ) d(x^2 + 1 )}{(x^2 + 1 )^2}\\
           &=& \frac{(x^2 + 1)2xdx - (x^2 - 1)2xdx}{(x^2 + 1)^2} = \frac{4x dx}{(x^2 + 1)^2},
\end{eqnarray*}

$$
\mbox{(c')}\; dz = (3)(\frac{5}{3})y^{2/3}dy= 5y^{2/3}dy.
$$
\noindent If we consider the composition of the function $y$ in (b) with the function $z$ in (c), we get for the differential of the composition

$$
dz = 5y^{2/3} \frac{4x dx}{(x^2 + 1)^2} = 5 \biggl(\frac{x^2 - 1}{x^2 + 1} \biggr)^{2/3} \frac{4x}{(x^2 + 1)^2} dx. 
$$
\end{example}

One traditional interpretation of the differential, which is especially useful in physics, is that of an ``infinitesimal." If $y = f(x)$, we know that $dy = f'(x) dx$. Now $dx$ is the function that assigns to every real number a the linear function defined by $(d_{a}x)(t) = x'(a)t = t$; i.e., it assigns the identity function. Hence we can interpret $dx$ as simply another independent variable. Then $dy$ is the variable whose value for a given $x$ and $dx$ is shown in Figure \f{2.21}. (Compare this illustration with Figure \f{2.20}.) The difference between the value of $f$ at $x$ and at $x + dx$ is denoted by $\Delta y$ in the figure. If $dx$ is chosen $y$-axis very small, then the difference between $dy$ and $\Delta y$ is relatively negligible. Hence $dy$ measures the resulting change in the value of $y = f(x)$ corresponding to an infinitesimal change $dx$ in the variable $x$.

\putfig{3.7truein}{scanfig2_21}{}{fig 2.21}

%EXAMPLE 4. 
\begin{example}

The height $h$ of a square pyramid is found to be 100 feet, and the length $x$ of one edge of its base is measured to be 160 feet. The volume $V$ of the pyramid is given by the formula $V = \frac{1}{3}hx^2$.  What error in the computed
%SEC. 6] THE DIFFERENTIAL  121
volume will result from an error of 4 inches in the measurement of $x$? If we consider $h$ as fixed, and $V$ as a function of $x$, then 

$$
dV = \frac{1}{3}h dx^2 = \frac{2}{3}hx dx.
$$
\noindent Since 4 inches is small compared with 160 feet, we set $dx$ = 4 inches = $\frac{1}{3}$ foot. The resulting change in volume is then approximately 

$$
dV = \frac{2}{3}(100)(160)\frac{1}{3} = \frac{32}{9} 1000 = 3555 \;\mbox{feet}^3.
$$
\noindent The percentage error in volume is

$$
 \frac{dV}{V} = \frac{ {\frac{2}{3}} hx dx}{{\frac{1}{3}}hx^2} = 2 \frac{dx}{x}.
$$
\noindent We compute $\frac{dx}{x} = \frac{\frac{1}{3}}{160} = 0.0021 = 0.21\%$, and so the percentage error in volume is only $0.42\%$.
\end{example}
