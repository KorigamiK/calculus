\section{Integrability of Monotonic Functions.}
 Let $f$ be a given function bounded on a closed interval $[a, b]$. How do we know whether or not $f$ is integrable over $[a, b]$, i.e., whether or not $\int_{a}^{b} f$ exists? In this section we shall give a partial answer, and also compute some integrals. Note that there is one situation where we know the answer immediately: If $a = b$, then all upper and lower sums are equal to zero. Hence $f$ is integrable, and

\begin{theorem} %(3.1)
$$
\int_{a}^{a} f = \int_{a}^{a}  f (x) dx = 0.  
$$
\end{theorem}

So we now assume that $a < b$. For every positive integer $n$, we shall denote by $\sigma_n$ the partition which subdivides $[a, b]$ into $n$ subintervals each of length $\frac{b - a}{n}$. Thus $\sigma_n = \{x_0,... , x_n
\}$, where

$$
x_i = a + \Bigl( \frac{b - a}{n} \Bigr)i,        \;\;\;     i = 0,..., n.
$$
\noindent Moreover,
$$
x_i - x_{i - 1} = \frac{b - a}{n},        \;\;\;     i = 1,..., n.
$$
%184 INTEGRATION [CHAP. 4
\noindent The upper and lower sums of $f$ relative to $\sigma_n$ will be denoted simply $U_n$, and $L_n$, respectively. That is, we abbreviate $U_{\sigma_n}$ by $U_n$, and in the same way $L_{\sigma_n}$ by $L_n$. One criterion for integrability is expressed in the following theorem.

\begin{theorem} %(3.2)
lf $\lim_{n \rightarrow \infty} (U_n - L_n) = 0$, then $f$ is integrable over $[a, b]$ and  
$$
\lim_{n \rightarrow \infty} L_n = \lim_{n \rightarrow \infty} U_n = \int_{a}^{b} f(x) dx.  
$$
\end{theorem}

\begin{proof}
We recall the basic theorem of Section 1---that the upper and lower sums of $f$ relative to any two partitions $\sigma$ and $\tau$ of the interval $[a, b]$ satisfy the inequality $L_\sigma \leq U_\tau$.  This implies, in particular, that any upper sum $U_{\tau}$ is an upper bound of the set \textbf{L} of all lower sums $L_{\sigma}$. Hence, by the Least Upper Bound Property, the set \textbf{L} has a least upper bound which we denote by $J$.  Since this number $J$ is an upper bound of \textbf{L}, we know that $L_\sigma \leq J$ for every partition $\sigma$.  Furthermore, since $J$ is a \textit{least} upper bound, we have $J \leq U_\tau$ for every partition $\tau$. Thus
$$
L_\sigma \leq J \leq U_\tau,
$$
for all partitions $\sigma$ and $\tau$ of $[a, b]$. As a special case of these inequalities, we have
\begin{equation}
L_{n} \leq J \leq U_{n}, \;\;\; \mbox{for every positive integer}\; n.  
\label{eq4.3.1}
\end{equation}
Since by hypothesis $\lim_{n \rightarrow \infty} (U_n - L_n) = 0$, it follows that this number $J$ is the only number which can lie between all upper and lower sums. Hence, by the definition, $f$ is integrable over $[a, b]$
and $J = \int_{a}^{b} f(x) dx$. From (1) we obtain the inequalities
$$
0 \leq J - L_n \leq U_n - L_n, \;\;\;\mbox{for every positive integer}\; n.
$$
Since the right side of the above inequalities approaches zero, the expression in the middle is caught in a squeeze and must also approach zero. Hence $\lim_{n \rightarrow \infty} (J - L_n) = 0$, or, equivalently,
$$
\lim_{n \rightarrow \infty} L_n = J = \int_{a}^{b} f(x) dx.
$$
Finally, consider the identity $U_n = J + (U_n - L_n) - (J - L_n)$. Since the two expressions in parentheses approach zero, it follows that
$$
\lim_{n \rightarrow \infty} U_n = J = \int_{a}^{b} f(x)dx, 
$$
and the proof is complete.
\end{proof}

An important class of functions to which the preceding theorem can be
readily applied, and which we now define, is the class of monotonic functions.
%SEC. 3] INTEGRABILITY OF MONOTONIC FUNCTIONS  185
To begin with, a real-valued function $f$ is said to be \textbf{increasing on an interval} $I$ if the domain of $f$ contains $I$ as a subset and if, for every $x_1$ and $x_2$ in $I$,

\begin{equation}
x_1 \leq x_2 \;\;\;\mbox{implies}\;\;\; f(x_1) \leq f(x_2).  
\label{eq4.3.2}
\end{equation}
\noindent If (2) holds for every $x_1$ and $x_2$ in the entire domain of $f$, we say simply that $f$ is an \textbf{increasing function}. Companion definitions are obtained by simultaneously replacing the second inequality in (2) by $f(x_1) \geq f(x_2)$ and the word \textbf{increasing} by the word \textbf{decreasing}. For example, the function $f$ defined by $f(x) = x^2$ is increasing on the interval $[0, \infty)$ and decreasing on the interval $(-\infty, 0]$. The function g defined by $g(x) = - 2x + 1$ is a decreasing function.

Note that, according to our definition, a constant function is both increasing and decreasing. Thus ``increasing," as it is used here, literally means ``nondecreasing," and in the same way ``decreasing" means ``nonincreasing."

A \textbf{monotonic function} is one which is either increasing or decreasing. Similarly, a function is \textbf{monotonic on an interval} if it is either increasing or decreasing on the interval. For such functions it is not difficult to prove the following integrability theorem.

% THEOREM. 
\begin{theorem} %(3.3)
If the function $f$ is monotonic on the closed interval $[a, b]$, then $f$ is integrable over $[a, b]$. Specifically, $\lim_{n \rightarrow \infty} (U_n - L_n) = 0$.
\end{theorem}

\begin{proof}
For the sake of concreteness, we shall assume that $f$ is increasing on $[a, b]$. An analogous argument works if $f$ is decreasing. By far the best proof of this theorem is obtained from a picture, which provides a completely convincing argument. A typical example of an increasing function together with a partition of the interval is shown in Figure \f{4.9}(a). The difference $U_n - L_n$ is equal to the sum of the areas of the shaded rectangles. By sliding these rectangles under one another to form a single stack, we obtain the tall rectangle shown in Figure \f{4.9}(b), whose area is also equal to $U_n - L_n$. 
This rectangle has base $\frac{b-a}{n}$ and altitude $f(b) - f(a)$. Its area is the product of these, and so
\putfig{4.5truein}{scanfig4_9}{}{fig 4.9}
$$
U_n - L_n = \Bigl( \frac{b - a}{n} \Bigr) (f(b) - f(a)).
$$
This difference can be made arbitrarily small by taking $n$ sufficiently large. It follows that $\lim_{n \rightarrow \infty} (U_n - L_n) = 0$, and we conclude from (3.2) that $f$ is integrable over $[a, b]$. This completes the proof.
\end{proof}


%EXAMPLE 1. 
\begin{example}
Evaluate $\int_{0}^{2} x^2 dx$. The function $f$ to be integrated is defined by $f(x) = x^2$, and the interval of integration is [0, 2]. Since $f$ is increasing on the interval, the integral certainly exists. The partition $\sigma_n = \{x_0, ..., x_n \}$ which subdivides [0, 2] into $n$ subintervals of equal length is given by

$$
x_i = a+ \frac{b -a}{n} i=0+\frac{2}{n} i  = \frac{2i}{n},
$$
\noindent for each $i = 0, . . ., n$. Moreover,  

$$
x_{i} - x_{i - 1} = \frac{b-a}{n} = \frac{2}{n}, \;\;\; i = 1, . . ., n.
$$
\noindent It follows from Theorems (3.2) and (3.3) that  
$$
\int_{0}^2 x^2 dx = \lim_{n \rightarrow \infty} U_n = \lim_{n \rightarrow \infty} L_n.
$$
\noindent That is, we may compute the integral using either the lower or the upper sums. Choosing the latter, we observe from Figure \f{4.10} that, on each subinterval $[x_{i - 1}, x_i]$, the function $f$ has its maximum value at the right endpoint, i.e., at $x_i$. Hence

$$
M_i = f(x_i),\;\;\;  i= 1,...,n.
$$
\noindent Since $f(x_i) = x_{i}^2$ and since $x_i = \frac{2i}{n}$, it follows that $M_i = x_{i}^2 = \frac{4i^2}{n^2}$. Substituting in the formula for the upper sum,  

$$
U_n = \sum_{i=1}^{n} M_i (x_i - x_{i - 1}),
$$
\noindent we obtain

$$
U_n = \sum_{i=1}^{n} \Bigl( \frac{4i^2}{n^2} \Bigr) \Bigl( \frac{2}{n} \Bigr) 
= \sum_{i=1}^{n} \frac{8i^2}{n^3} = \frac{8}{n^3} \sum_{i=1}^{n} i^2.
$$
%SEC. 3] ~NTEGRAH]LITY OF MONOTON!C FUNCT!ONS 187
\noindent From (2.5), we have

$$
\sum_{i=1}^{n} i^2 = \frac{n(n + 1)(2n + 1)}{6} = \frac{ 2n^3 + 3n^2 + n}{6}.
$$
\noindent Hence
$$
U_n = \frac{8}{n^3} \frac{2n^3 + 3n^2 + n}{6} = \frac{4}{3} \Bigl( 2 + \frac{3}{n} + \frac{1}{n^2} \Bigr),
$$
\noindent and so

$$
\lim_{n \rightarrow \infty} U_n = \lim_{n \rightarrow \infty} \frac{4}{3} \Bigl( 2 + \frac{3}{n} + \frac{1}{n^2} \Bigr)
=
\frac{4}{3} \cdot  2 = \frac {8}{3}.
$$
\noindent We conclude that
$$
\int_{0}^{2} x^2 dx = \frac{8}{3}.
$$
\end{example}
\medskip

\putfig{3truein}{scanfig4_10}{}{fig 4.10}

It was shown in Section 1 that the integral of a nonnegative function is equal to the area under its graph. It follows from the above example that the area of the region bounded by the parabola $y = x^2$, the $x$-axis, and the line $x = 2$ is equal to $\frac{8}{3}$.

%EXAMPLE 2
\begin{example}
Evaluate $\int_{1}^{4} (5 - x) dx$. The function $f$, defined by $f(x) = 5 - x$, is linear and decreasing on the interval [1, 4]. Its graph is shown in Figure \f{4.11}. The partition $\sigma_{n} = \{ x_{0}, . . ., x_{n} \}$ subdivides the interval [1, 4] into subintervals of length $\frac{4 - 1}{n} = \frac{3}{n}$, and the points are given by  

$$
x_{i} = 1 + \Bigl(\frac{3}{n}\Bigr) i, \;\;\;  i = 0, . . . , n.
$$
%188 INTEGRATION [CHAP. 4 
\noindent In addition,
$$
x_{i} - x_{i-1} = \frac{3}{n}, \;\;\;  i = 1, ... , n.
$$
\noindent We shall compute the integral as a limit of lower sums, and it follows from Theorems (3.2) and (3.3) that

$$
\int_{1}^{4} (5 - x)dx = \lim_{n \rightarrow \infty} L_n.
$$
\putfig{3truein}{scanfig4_11}{}{fig 4.11}

\noindent Since $f$ is decreasing, its minimum value on each subinterval $[x_{i-1}, x_i]$ occurs at the right endpoint. Hence

$$
m_i = f(x_i),           \;\;\;   i= 1,...,n.
$$
\noindent We have $x_i = 1 +\frac{3i}{n}$ and $f(x_i) = 5 - x_i$, and so 

$$
m_i= 5 - \Bigl( 1 + \frac{3i}{n} \Bigr)  = 4 - \frac{3i}{n}.
$$
\noindent Since $x_{i} - x_{i-1} = \frac{3}{n}$, we get 

$$
L_n = \sum_{i=1}^{n} m_{i}(x_{i} - x_{i-1}) = \sum_{i=1}^{n} \Bigl( 4 - \frac{3i}{n} \Bigr) \frac{3}{n}.
$$
% SEC. 3] TNTEGRABILITY OF MONOTONIC FUNCTIONS  189
\noindent The rest of the problem uses the manipulative techniques of the summation convention.

\begin{eqnarray*}
L_n &=& \sum_{i=1}^{n} \Bigl( 4 - \frac{3i}{n} \Bigr) \frac{3}{n} =  \sum_{i=1}^{n} \Bigl( \frac{12}{n} - \frac{9i}{n^2} \Bigr)\\
       &=& \sum_{i=1}^{n} \frac{12}{n} -  \sum_{i=1}^{n} \frac{9i}{n^2} \\
       &=& \frac{12}{n} \sum_{i=1}^{n} 1 - \frac{9}{n^2} \sum_{i=1}^{n} i.
\end{eqnarray*}
\noindent since $\sum_{i=1}^{n} 1 = n$ and since $\sum_{i=1}^{n}  i = \frac{n(n + 1)}{2}$, we get 
\begin{eqnarray*}
L_n &=& \frac{12}{n} - \frac{9}{n^2} \frac{ n(n+ 1)}{2} \\
       &=&12 - \frac{9}{2} \Bigl(1 + \frac{1}{n} \Bigr) .
\end{eqnarray*}
\noindent But it is easy to see that 
$$
\lim_{n \rightarrow \infty} \Bigl [12 - \frac{9}{2} \Bigl( 1 + \frac{1}{n} \Bigr) \Bigr] = 12 - \frac{9}{2} = 7
\frac{1}{2},
$$
\noindent and we finally conclude that

$$
\int_{1}^{4} (5 - x) dx = \lim_{n \rightarrow \infty} L_n = 7 \frac{1}{2}.
$$
\noindent This answer can be checked by looking at Figure \f{4.11}. The value of the integral is equal to the area of the shaded region $P$, which is divided by the horizontal line $y = 1$ into two pieces: a right triangle sitting on top of a rectangle. The area of the triangle is $\frac{1}{2} (3 \cdot 3) = \frac{9}{2}$, 
and that of the rectangle is $3 \cdot 1 = 3$. Hence  
$$
\int_{1}^{4} (5 - x) dx = area(P) = \frac{9}{2} + 3 = 7\frac{1}{2}.
$$
\end{example}
\medskip

The excessive lengths of the computations in Examples 1 and 2 make it obvious that some powerful techniques are needed to streamline the process of evaluating definite integrals. The advent of modern high-speed computers is one answer to the problem, and occasionally, as in Example 2, a simple formula for area will do the trick. The classical solution to the problem,
however, is the Fundamental Theorem of Calculus, which we shall study in detail in Section 5.
%INTEGRATION [CHAP. 4

