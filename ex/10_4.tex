\begin{exercises}

\ex{10.4.1}
For each of the following parametrizations and
values of $t_0$, compute $P(t_0)$ and
the derived vector $\vec dP(t_0)$.  Draw
the parametrized curve and each of the
tangent vectors $\vec dP(t_0)$ to the curve.
\begin{exenum}
\x
$P(t) = (x(t),y(t)) = (t-1,t^2), \quad 
-\infty < t < \infty$;

$t_0 = -1$, $t_0 = 0$, and $t_0 = 2$.
\x
$P(t) = (x(t),y(t)) = (t^2+1,t-1), \quad 
-\infty < t < \infty$;

$t_0 = -1$, $t_0 = 0$, and $t_0 = 1$.
\x
$P(t) = (t-1,t^3), \quad 
-\infty < t < \infty$;

$t_0 = 0$, $t_0 = 1$, and $t_0 = 2$.
\x
$P(t) = (x,y) = (e^t,t), \quad 
-\infty < t < \infty$;

$t_0 = 0$ and $t_0 = \ln2$.
\x
$P(t) = (3\cos t,2\sin t), \quad 
-\infty < t < \infty$;

$t_0 = 0$, $t_0 = \frac\pi4$, and $t_0 = \frac\pi2$.
\x
$P(t) = (x(t),y(t)) = (t-1,t^2), \quad 
-\infty < t < \infty$;

$t_0 = -1$, $t_0 = 0$, and $t_0 = 2$.
\x
$P(t) = (t^2,t^3), \quad 
-\infty < t < \infty$;

$t_0 = -1$, $t_0 = 0$, and $t_0 = 2$.
\x
$P(t) = (t-1,2t+4), \quad 
-2 \leq t \leq 2$;

$t_0 = -1$, $t_0 = 0$, and $t_0 = 1$.
\end{exenum}

\ex{10.4.2}
For each of the following parametrizations
$P(t) = (x(t),y(t))$, find the derived vector
$\vec dP(t)$ for an arbitrary value of $t$ in the
domain.  Draw the vectors $\vec dP(0)$,
$\vec dP(1)$, and $\vec dP(2)$ in the $xy$-plane.
\begin{exenum}
\x
$\dilemma{x(t) = t^2-1,}
{y(t) = t^3, \quad -1\leq t \leq3.}$
\x
$\dilemma{x(t) = \frac12(e^t+e^{-t}),}
{y(t) = \frac12(e^t-e^{-t}), \quad
-\infty < t < \infty.}$
\x
$\dilemma{x(t) = t^2,}
{y(t) = \frac23(3t+1)^{\frac32}, \quad
-\frac13 \leq t \leq 5.}$
\x
$\dilemma{x(t) = t^2+t+1,}
{y(t) = \frac{t^3}3 + t^2 - 1, \quad
-\infty < t < \infty.}$
\end{exenum}

\ex{10.4.3}
The cycloid shown in Figure \figref{10.4} is
defined by a parametrization $P(\theta) = (x,y)$
in which
\[
\dilemma{x=a(\theta - \sin \theta),}
{y=a(1-\cos\theta), \quad
-\infty < t < \infty.}
\]
Compute the derived vector $\vec dP(\theta)$.
Sketch the curve, and draw the tangent vectors
$\vec dP(0)$, $\vec dP\left(\frac\pi2\right)$,
$\vec dP(\pi)$, and $\vec dP(2\pi)$.

\ex{10.4.4}
Prove that the curve defined parametrically by the
equations
\[
\dilemma{x = sd_1 + a,}
{y = sd_2 + b, \quad -\infty < s < \infty,}
\]
where not both $d_1$ and $d_2$ are zero,
is a straight line.
(\emph{Note:} Check the definition of a straight
line given in section \secref{1.5}.)

\ex{10.4.5}
Converse of Problem \exref{10.4.4}:
Prove that, if $L$ is a straight line in $\R^2$,
then it can be defined by a parametrization
$P(s) = (x,y)$ for which
\[
\dilemma{x = sd_1 + a,}
{y = sd_2 + b, \quad -\infty < s < \infty,}
\]
and not both $d_1$ and $d_2$ are zero.

\ex{10.4.6}
For each of the following parametrizations
$P(t) = (x(t),y(t))$ and values of $t_0$,
compute the derived vector $\vec dP(t_0)$.
Draw the parametrized curve, the tangent
line at $t_0$, and write an equation in $x$
and $y$ of the tangent line.
\begin{exenum}
\x
$P(t) = (t^2+1, t+1)$, $-\infty < t < \infty$,
and $t_0 = 2$.
\x
$P(t) = (t^2+1, t+1)$, $-\infty < t < \infty$,
and $t_0 = 0$.
\x
$P(t) = (e^t,t)$, $-\infty <t<\infty$,
and $t_0 = \ln 2$.
\x
$P(t) = (|t|, t)$, $-\infty < t < \infty$,
and $t_0 = 0$.
\end{exenum}

\ex{10.4.7}
Let $P$ be the parametrization defined by
$P(t) = (t^2, \frac12t^2)$, for every real number $t$.
\begin{exenum}
\x
Write an equation in $x$ and $y$ of the tangent line
at $t=2$.
\x
Describe the vector space of tangent vectors
at $t=2$ and at $t=0$.
\end{exenum}

\ex{10.4.8}
Let $f$ be a real-valued function which is
differentiable at $a$.
\begin{exenum}
\x
Write an equation of the line tangent to the graph
of $f$ at $(a,f(a))$.
\x
Consider the parametrization
\[
P(t) = (t,f(t))
.
\]
Compute the derived vector $\vec dP(a)$,
and write an equation of the tangent line
to the parametrized curve at $a$.
\end{exenum}

\ex{10.4.9}
Let $P:[a,b]\goesto\R^2$ be a parametrization
for which the derivatives $x^\prime$ and $y^\prime$
of the coordinate functions are continuous.
Prove that the arc length of the curve
parametrized by $P$ is given by
\[
{L_a}^b = \int_a^b |\vec dP(t)| \; dt
.
\]

\end{exercises}
