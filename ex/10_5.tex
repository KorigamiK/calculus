\begin{exercises}

\ex{10.5.1}
\begin{exenum}
\x
\xlab{10.5.1a}
Draw each of the following vectors.

(i) $(0,5)_{P_0}$, where $P_0 = (-1,1)$.

(ii) $(4,-1)_{P_1}$, where $P_1 = (1,-1)$.

(iii) $(1,3)_{P_2}$, where $P_2 = (1,1)$.

(iv) $(-2,-3)_{P_3}$, where $P_3 = (0,0)$.
\x
Let $P_0 = (-1,1)$, and compute and draw the
translated vectors $T_{P_0}(\vec u)$,
where $\vec u$ is taken to be each of the four
vectors in \exref{10.5.1a}.
\end{exenum}

\ex{10.5.2}
A particle moves in the plane during the time interval
from $t=0$ to $t=2$ seconds.
Its position at any time during this interval is given
by the parametrization
\[
P(t) = (t,t^2-t)
,
\]
where it will be assumed that the unit of
distance in the plane is $1$ foot.
\begin{exenum}
\x
\xlab{10.5.2a}
Identify and draw the curve which the particle traces
out during its interval of motion.
\x
Compute the velocity vector $\vec v(t)$.
Find the position, velocity, and speed at
$t=0$, $t=1$, and $t=2$.
Show these positions and draw the velocity
vectors in the figure in part \exref{10.5.2a}.
\x
Compute the acceleration $\vec a(t)$.
Find the times and corresponding positions
(if any) when the acceleration and velocity vectors
are perpendicular to each other.
\x
\xlab{10.5.2d}
Write a definite integral equal to the distance (in feet)
which the particle moves during the interval
from $t=0$ to $t=2$ seconds.
\x
Evaluate the integral in \exref{10.5.2d}.
\end{exenum}

\ex{10.5.3}
An object is dropped from an airplane which is
flying in a straight line over level ground at a constant
speed of $800$ feet per second and at an
altitude of $10,000$ feet.
The horizontal coordinate of the velocity of the object
is constant and equal in magnitude to the speed
of the plane.
The vertical coordinate of velocity is initially zero.
However, the vertical component of acceleration
(due to gravity) is $-32$ feet per second per second.
(These data are realistic only if we neglect air
resistance, the curvature of the earth, etc.)
\begin{exenum}
\x
Define a parametrization $P(t)=(x(t),y(t))$ which gives
the position of the particle at time $t$.
Assume that the object was dropped when $t=0$
and that $P(0) = (0,0)$.
Compute the velocity and acceleration vectors
$\vec v(t)$ and $\vec a(t)$.
\x
How long does it take the object to fall to the ground?
\x
Identify and draw the curve in which the object falls.
\x
Express the distance traveled along the curve as
a definite integral.
\end{exenum}

\ex{10.5.4}
Consider a particle in motion in the plane from
$t=0$ to $t=4$ seconds.  Its position at any time
during this interval is given by
\[
P(t) = (x,y) = \left((t-2)^2, (t-2)^2\right)
,
\]
where it is assumed that the unit of distance
in the plane is $1$ foot.
\begin{exenum}
\x
Draw the curve in which the particle moves
during the interval.
\x
Complete the velocity $\vec v(t)$ and the speed
$|\vec v(t)|$.
What are the minimum and maximum speeds,
and at what times are they attained?
\x
Describe the vector space of tangent vectors to the
parametrized curve at $t=1$, and also at $t=2$.
\x
Compute the distance traveled by the particle during
the motion.
\end{exenum}

\ex{10.5.5}
The position of a particle in motion in the plane
is defined by the parametrization:
\[
P(t) = (x,y) = (t^2,t^3), \quad -2 \leq t \leq 2
.
\]
\begin{exenum}
\x
\xlab{10.5.5a}
Draw the curve traced out by the particle during
the interval $[-2,2]$.
\x
Compute the velocity vector $\vec v(t)$.
Find the position, velocity, and speed at
$t=-2$, $t=0$, $t=1$, and $t=2$.
Indicate these positions and draw the
velocity vectors in the figure in \exref{10.5.5a}.
\x
Compute the accleration vector $\vec a(t)$.
Determine the four specific vectors
$\vec a(-2)$, $\vec a(0)$, $\vec a(1)$, and $\vec a(2)$,
and draw them in the figure in \exref{10.5.5a}.
\end{exenum}

\ex{10.5.6}
The position of a particle in the plane is defined by the
parametrization
\[
\dilemma{x = a\cos kt,}
{y = b \sin kt, \quad -\infty < t < \infty,}
\]
where $a$, $b$, and $k$ are positive constants
and $a > b$.
\begin{exenum}
\x
Identify and draw the curve in which the particle moves.
\x
Prove that the particle is never at rest.
\x
Show that the acceleration vector $\vec a(t)$
always points directly toward the origin.
\end{exenum}

\ex{10.5.7}
Prove that parallel translation has the following properties:
\begin{exenum}
\x
$|T_{P_0} (\vec u)| = |\vec u|$
for every vector $\vec u$.
\x
If $\vec u$ is any vector in $V_{P_0}$,
then $T_{P_0}(\vec u) = \vec u$.
\x
If $\vec 0$ is any zero vector,
then $T_{P_0}(\vec 0)$ is also a zero vector.
\end{exenum}

\ex{10.5.8}
Starting at $t=0$, a stone at the end of a string
is whirled around in a fixed circle of radius $a$
at ever-increasing speed equal to $kt$ for some
positive constant $k$.
The tension in the string is equal to $m|\vec a(t)|$,
where $m$ is the mass of the stone and $|\vec a(t)|$
is the length of the acceleration vector.
Suppose the string breaks when the tension exceeds
some value $T$.
Compute, in terms of the constants $a$, $k$, $m$,
and $T$, the moment when the string breaks.

\end{exercises}
