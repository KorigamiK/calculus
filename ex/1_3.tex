\begin{exercises}

\ex{1.3.1}
Let functions $f$ and $g$ be defined by
\[
f(x)=x^3-4x^2+5x-2= (x-2)(x^2-2x+1),\ g(x) = \frac1x
.
\]
Find $h(x)$ if
\begin{exenum}
\sx
$h=f(g)$
\sx
$h=f+g$
\sx
$h=g(f)$
\sx
$h=fg$
\sx
$h=5fg^2$.
\end{exenum}

\ex{1.3.2}
What is the domain and range of the functions $f$ and $g$
in Problem \exref{1.3.1}?
What is the domain of each of the functions $h$?

\ex{1.3.3}
If $f(x)=x+1$ and $g(x) = x-1$, plot the graph of the function
$\frac fg$.

\ex{1.3.4}
Plot the graph of the composite function $F(g)$, where
$F$ and $g$ are the functions defined by
$g(x) = x-2$ and $F(x)= \frac1x$.

\ex{1.3.5}
If $f$, $g$, and $h$ are functions, show that $f(g(h)) = (f(g))(h)$.
This is the Associative Law for the Composition of Functions.

\ex{1.3.6}
If $f$ is a real-valued function, how would you define the functions $3f$?
How would you define $\sqrt f$?

\ex{1.3.7}
The velocity $v$ of a freely falling body depends on the distance
$s$ that it has fallen according to the equation $v = \sqrt{2gs}$,
where $g$ is the constant gravitational acceleration.
\begin{exenum}
\sx
\sxlab{1.3.7a}
Using an $s$-axis and a $v$-axis, plot the dependent variable $v$
as a function of the independent variable $s$.
\sx
\sxlab{1.3.7b}
If $s$ depends on the time $t$ according to the equation
$s=\frac12gt^2$, how does $v$ depend on $t$?
\end{exenum}
Note that the variable $v$ in \exref{1.3.7a},
which depends on $s$, is not the same function as the
variable $v$ in \exref{1.3.7b}, which depends on $t$.
Without knowing which is referred to, the meaning of the value of
$v$ at 2 is ambiguous.

\ex{1.3.8}
If $w = u^2 + u + 1$, $u = x^2 + 2$, and $v = x - 1$, what is the value
of each of the following functions at an arbitrary real number $x$?
\begin{exenum}
\sx
$u + v$
\sx
$w + v$
\sx
$wu$.
\end{exenum}

\ex{1.3.9}
If $F(x) = x^3 + x + 2$ and $u = x^2 +1$ and $w = \frac{x+1}x$, then
\begin{exenum}
\sx
$(F(u))(x)=$
\sx
$F(w(x))=$
\sx
$(u+v)(x)=$
\end{exenum}

\ex{1.3.10}
The equation $y=2x +1$ defines $y$ as a function of $x$.
It also defines $x$ as a function of $y$.
Describe the latter function in two ways.

\ex{1.3.11}
Draw the graph of the function $f(x)=ax-1$ for four different values
of the constant $a$.

\ex{1.3.12}
If $f$ and $g$ are two real-valued functions, give the definitions 
of the sum $f+g$ and the product $fg$ in terms of ordered pairs.

\ex{1.3.13}
Let $f$ and $g$ be two real-valued functions.
In terms of domain $f$ and domain $g$, what are:
\begin{exenum}
\sx
domain $f(g)$?
\sx
domain $(f+g)$?
\sx
domain $fg$?
\end{exenum}

\end{exercises}
