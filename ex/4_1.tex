\begin{exercises}

\ex{4.1.1}
Draw the graph of the function $f$ defined by
$f(x) = \frac1x$,
and answer the following questions.
\begin{exenum}
\x
Is $f$ bounded on the closed interval $[2,5]$?
\x
Is $f$ bounded on the open interval $(2,5)$?
\x
Does $f$ have an upper bound on the interval $(0,2)$?
If so, give one.
\x
Does $f$ have a lower bound on the interval $(0,2)$?
If so, give one.
\end{exenum}

\ex{4.1.2}
If the number $M$ is the least upper bound of the set
of all numbers $f(x)$ for $x$ lying in an interval $I$,
we say simply that $M$ is the least upper bound
of $f$ on $I$.
A similar remark holds for the greatest lower bound.
Draw the graph of the function $f$ defined by
$f(x) = \frac1{x-1}$, and answer the following questions.
\begin{exenum}
\x
What is the least upper bound of $f$ on the closed interval
$[2,3]$?
\x
What is the greatest lower bound of $f$ on $[2,3]$?
\x
What are the least upper bound and greatest lower bound
of $f$ on the open interval $(2,3)$?
\x
What is the greatest lower bound of $f$ on the interval
$(1,2)$?
\end{exenum}

\ex{4.1.3}
Compute the upper and lower sums $U_\sigma$ and $L_\sigma$
in each of the following examples.
\begin{exenum}
\x
$f(x) = \frac1x$, $[a,b] = [1,4]$, and
$\sigma = \{1,2,3,4\}$.
\x
$f(x) = \frac x2$, $[a,b] = [0,2]$, and
$\sigma = \{0, \frac13, \frac23, 1, \frac43, \frac53, 2\}$.
\x
$g(x) = x^2 + 1$, $[a,b] = [0,1]$, and
$\sigma = \{x_0,x_1,x_2,x_3,x_4,x_5\}$,
where $x_i = \frac i5, i = 0, \ldots, 5$.
\x
$g(x) = x^3$, $[a,b] = [-1,1]$, and
$\sigma = \{-1, -\frac12, 0, \frac12, 1\}$.
\end{exenum}

\ex{4.1.4}
True or false, and give your reason:
If a function $f$ is continuous at every $x$ is a
closed interval $[a,b]$, then $f$ has both a least upper
bound and a greatest lower bound on $[a,b]$.

\ex{4.1.5}
Assume that the function $f$ defined by
$f(x) = x^2 + 1$ is integrable over the interval $[0,1]$.
Using the partition
$\sigma = \{0, \frac15, \frac25, \frac35, \frac45, 1\}$,
show that
\[
\frac{31}{25} \leq \int_0^1 f \leq \frac{36}{25}
.
\]

\ex{4.1.6}
Assuming that the function $g$ defined by $g(x) = 2x$
is integrable over the interval $[0,2]$, use the partition
$\sigma = \{0, \frac12, 1, \frac32, 2\}$ to show that
\[
3 \leq \int_0^2 2x\;dx \leq 5
.
\]

\ex{4.1.7}
Assume that the function $x^2$ is integrable over the
interval $[0,1]$.  Using the partition
$\sigma = \{x_0, \ldots, x_n\}$, where $n=10$
and $x_i = \frac{i}{10}$, for $0,\ldots,10$,
prove that
\[
\frac{57}{200} \leq \int_0^1 x^2 \; dx \leq \frac{77}{200}
.
\]

\ex{4.1.8}
Compute the definite integral
$\int_a^b f = \int_a^b f(x) \; dx = \int_a^b f(t) \; dt$
in each of the following examples.
Assume that $f$ is integrable, and use
Theorem \thref{4.1.4} and the standard formulas
for the areas of simple plane figures.
In each case,
draw the graph of $f$ and shade the region $P$.
\begin{exenum}
\x
$\int_{-1}^1 f$, where $f(x) = \sqrt{1-x^2}$.
\x
$\int_1^2 f(t) \; dt$, where $f(t) = t-1$.
\x
$\int_0^2 2x \; dx$
\x
$\int_0^1 (5-2x) \; dx$
\x
$\int_{-1}^1 |x| \; dx$.
\end{exenum}

\ex{4.1.9}
It is stated in this section that the first condition for
integrability is always satisfied:
If $f$ is bounded on $[a,b]$, then there exists a real
number $J$ such that $L_\sigma \leq J \leq U_\tau$
for any two partitions $\sigma$ and $\tau$ of $[a,b]$.
\begin{exenum}
\x
Show that one such number is the least upper bound
of all the lower sums $L_\sigma$.
(This number is called the \dt{lower integral of $f$
from $a$ to $b$}.)
\x
Show that another possibility is the greatest lower bound
of all the upper sums $U_\tau$.
(This number is the \dt{upper integral of $f$ from
$a$ to $b$}.)
\x
Show that $f$ is integrable over $[a,b]$ if and only
if the lower integral from $a$ to $b$ equals the upper integral,
and that if the lower integral equals the upper then their
common value is $\int_a^b f$.
\end{exenum}

\end{exercises}
