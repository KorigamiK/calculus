\begin{exercises}

\ex{4.7.1}
For each of the following functions and intervals:
Compute $\int_a^b f(x) \; dx$;
draw the graph of $f$;
label the region $P^-$ on or below the $x$-axis
which is bounded by the curve $y = f(x)$,
the $x$-axis, and the lines $x = a$ and $x = b$;
label the analogous region $P^-$ on or below
the $x$-axis;
evaluate $\mbox{\emph{area}}(P^+)$ and
$\mbox{\emph{area}}(P^-)$;
and check formula \eqref{4.7.1}.
\begin{exenum}
\x
$f(x) = x - 1$, $a = 0$, and $b = 4$.
\x
$f(x) = -x^2 + x + 2$, $a = 0$, and $b = 3$.
\x
$f(x) = -x^2 + x + 2$, $a = -2$, and $b = 3$.
\x
$f(x) = (x - 1)^3$, $a = 0$, and $b = 2$.
\end{exenum}

\ex{4.7.2}
In each of the following find the area of the subset
$P^+ \cup P^-$ of the $xy$-plane bounded by the curve
$y = f(x)$, the $x$-axis, and the lines $x = a$ and $x = b$.
\begin{exenum}
\x
$f(x) = x^5$, $a = -1$, and $b = 1$.
\x
$f(x) = x^2 - 3x + 2$, $a = 0$, and $b = 2$.
\x
$f(x) = (x + 1)(x - 1)(x - 3)$, $a = 0$, and $b = 2$.
\x
$f(x) = |x^2 - 1|$, $a = -2$, and $b = 2$.
\end{exenum}

\ex{4.7.3}
Let $f$ be a continuous function.
Using areas, show that
\begin{exenum}
\x
\xlab{4.7.3a}
If $f$ is an odd function, then
$\int_{-a}^a f(x) \; dx = 0$.
\x
\xlab{4.7.3b}
If $f$ is an even function, then
$\int_{-a}^a f(x) \; dx = 2 \int_0^a f(x) \; dx$.
\end{exenum}

\ex{4.7.4}
Prove \exref{4.7.3a} and \exref{4.7.3b}
analytically using the Fundamental Theorem of Calculus.
[More specifically, use Theorems \thref{4.5.2}
and \thref{4.5.3}.]

\ex{4.7.5}
Draw the region $R$ bounded by the lines
$x = 0$ and $x = 2$ and lying between the
graphs of the functions $f(x) = x + 2$
and $g(x) = (x - 1)^2$.
Find the area of $R$.

\ex{4.7.6}
Draw the region $Q$ lying to the right of the $y$-axis
and bounded by the curves $x = 0$, $3y - x + 3 = 0$,
and $3y + 3x^2 - 8x = 3$.
Compute $\mbox{\emph{area}}(Q)$.

\ex{4.7.7}
Find the area of the subset $R$ of the $xy$-plane
lying between the lines $x = \frac12$ and $x = 2$,
and between the graphs of the functions
$f(x) = \frac1{x^2}$ and $g(x) = x^2$.
Draw the relevant lines and curves and indicate
the region $R$.

\ex{4.7.8}
Find the area of the region bounded by the two parabolas
$y = -x^2 + x + 2$ and $y = x^2 - 2x$.

\ex{4.7.9}
Draw the graphs of the equations $y = x^2$
and $y = 4$, and label the region $R$ bounded by them.
\begin{exenum}
\x
\xlab{4.7.9a}
Express the area of $R$ as an integral with respect
to $x$ using \thref{4.7.1}.
Evaluate the integral.
\x
Similarly, express the area of $R$ as an integral with
respect to $y$ using the counterpart of \thref{4.7.1}
for functions of $y$.  Evaluate the integral and
check the answer to \exref{4.7.9a}.
\end{exenum}

\ex{4.7.10}
\begin{exenum}
\x
If $f(y) = -y^2 + y + 2$, sketch the region bounded
by the curve $x = f(y)$, the $y$-axis,
and the lines $y = 0$ and $y = 1$.
Find its area.
\x
Find the area bounded by the curve
$x = -y^2 + y + 2$ and the $y$-axis.
\x
The equation $x + y^2 = 4$ can be solved for $x$
as a function of $y$, or for $y$ as plus or minus a
function of $x$.  Sketch the region in the first quadrant
bounded by the curve $x + y^2 = 4$, and find its area
first by integrating a function of $y$ and then by
integrating a function of $x$.
\end{exenum}

\ex{4.7.11}
If the function $f$ is continuous at every point of the
interval $[a,b]$ and may cross the $x$-axis at a
finite number of points in the interval.
Let $P^+$ and $P^-$ have their usual meaning
[as in formula \eqref{4.7.1}].
\begin{exenum}
\x
Is $|f(x)|$ continuous at every point of $[a,b]$?
\x
Show that
\[
\mbox{\emph{area}}(P^+ \cup P^-) =
\int_a^b |f(x)| \; dx
.
\]
\end{exenum}

\ex{4.7.12}
Find the area of the region bounded by the parabola
$y = x^2$, the $x$-axis, and the line tangent
to the parabola at the point $(2, 4)$.
Do the problem
\begin{exenum}
\x
using $x$ as the variable of integration.
\x
using $y$ as the variable of integration.
\end{exenum}

\ex{4.7.13}
Do Problem \exref{4.7.12} for the line tangent to the
parabola at the general point $(a, a^2)$.

\ex{4.7.14}
Express the area of the ellipse
$\frac{x^2}{a^2} + \frac{y^2}{b^2} = 1$
as a definite integral of a function of $x$,
and as a definite integral of a function of $y$.
(The resulting indefinite integrals cannot
be evaluated with the theory so far developed.)

\ex{4.7.15}
Find the area of the shaded region in Figure \figref{4.23}.
The curves are parabolas.  The inscribed square
has area $4$, and the circumscribed square has
area $16$.

\end{exercises}
