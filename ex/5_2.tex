\begin{exercises}

\ex{5.2.1}
One must be careful to distinguish between the
inverse of a function and the reciprocal of the function.
If $f(x) = 3x + 4$, write
\begin{exenum}
\x
$f^{-1}(x)$, the value of the inverse function
$f^{-1}$ at $x$.
\x
$[f(x)]^{-1}$, the reciprocal of $f(x)$.
\end{exenum}

\ex{5.2.2}
If $f$ has an inverse $f^{-1}$,
what relations hold between the domains and ranges
of $f$ and $f^{-1}$?

\ex{5.2.3}
Find the derivative with respect to $x$
of each of the following functions.
\begin{exenum}
\x
$e^{7x}$
\x
$\frac13e^{3x+2}$
\x
$xe^x$
\x
$e^{-x^2}$
\x
$e^x \ln x$
\x
$\frac{e^x+e^-x}2$
\x
$\frac{e^x-e^-x}2$
\x
$\frac3{2e^x}$
\x
$\frac{e^x}x$
\x
$\frac{5\ln x}x$
\x
$e^{3x^2 - 4x + 5}$
\x
$e^{ax+b}$.
\end{exenum}

\ex{5.2.4}
Solve each of the following integrals.
\begin{exenum}
\x
$\int e^{3x} \;dx$
\x
$\int 2xe^{x^2} \;dx$
\x
$\int \frac1x e^{\ln x} \;dx$
\x
$\int \frac{e^x + e^{-x}}{e^x - e^{-x}} \;dx$
\x
$\int \frac{xe^{x^2} \;dx}{4e^{x^2} + 5}$
\x
$\int \frac {3\;dx}{2e^{4x}}$
\x
$\int \frac {x^2\;dx}{e^{x^3-2}}$
\x
$\int (x+1) e^{x^2+2x} \; dx$
\x
$\int e^\pi \;dx$
\x
$\int e^{ax+b} \;dx$.
\end{exenum}

\ex{5.2.5}
Evaluate each of the following integrals.
\begin{exenum}
\x
$\int_1^2 \frac{dx}x$
\x
$\int_0^3 e^{2x} \;dx$
\x
$\int_x^{x^2} e^t \; dt$
\x
$\int_4^8 \frac{dx}{e^x}$.
\end{exenum}

\ex{5.2.6}
Sketch the graph of each of the following equations.
Label all extreme points and point of inflection,
and give the values of $x$ at which these occur.
Classify each extreme point as a local
maximum or minimum.
\begin{exenum}
\x
$y=e^{3x}$
\x
$y=x \ln x$
\x
$y = xe^{-x}$
\x
$y = x^2e^{-x}$
\x
$y = e^{-x^2}$.
\end{exenum}

\ex{5.2.7}
In each of the following examples,
find the area of the region above the $x$-axis,
below the graph of the function $f$,
and between two vertical lines whose equations
are given.
\begin{exenum}
\x
$f(x) = 2e^{4x}$, $x=0$ and $x=2$.
\x
$f(x) = xe^{x^2}$, $x=2$ and $x=4$.
\x
$f(x) = \frac1{e^x}$, $x=5$ and $x=7$.
\end{exenum}

\ex{5.2.8}
Suppose $f$ is a function which has the property
that it is equal to its own derivative;
i.e., $f^\prime = f$.
\begin{exenum}
\x
\xlab{5.2.8a}
Compute the derivative of the quotient
$\frac{f(x)}{e^x}$.
\x
Using the result of \exref{5.2.8a},
prove that $f(x) = ke^x$ for some constant $k$.
\end{exenum}

\ex{5.2.9}
Let $f$ be a function with domain
$[0,1]$ and defined by
\[
f(x) = \sqrt{2x-x^2}, \quad 0 \leq x \leq 1
.
\]
Draw the graph of $f$ and the graph of the
inverse function $f^{-1}$.

\ex{5.2.10}
Compute each of the following limits using
L'H\^opital's Rule or some other method if you prefer.
\begin{exenum}
\x
$\lim_{x\goesto0} \frac{e^x-1}x$
\x
$\lim_{x\goesto0} \frac{e^x-1-x}{x^2}$
\x
$\lim_{x\goesto1} \frac{e^x-e}{x-1}$
\x
$\lim_{x\goesto0} \frac{x^2e^x}{1-e^{x^2}}$
\x
$\lim_{x\goesto\infty} \frac{x}{e^x}$
\x
$\lim_{x\goesto\infty} \frac{e^x}{x^6}$
\x
$\lim_{x\goesto0+} \frac{\ln x}{e^\frac1x}$
\x
$\lim_{x\goesto0} \frac{e^x-e^{-x}}x$.
\end{exenum}

\ex{5.2.11}
Show by means of an example that half of
\thref{5.2.3} is not enough.
That is, define two functions $f$ and $g$
such that $f(g(x)) = x$, for every $x$ is the
domain of $g$, but $g \ne f^{-1}$.

\ex{5.2.12}
Prove that $\lim_{x\goesto0} \frac{e^x-1}x = 1$,
using $\frac{d}{dx} e^x = e^x$
and the definition of the derivative at $0$.

\ex{5.2.13}
A function of $x$ is a solution of a differential
equation if it and its derivatives make the equation true.
For what value (or values) of $m$ is $y = e^{mx}$
a solution of
$\frac{d^2y}{dx^2} - 3 \frac{dy}{dx} + 2y = 0$?

\ex{5.2.14}
Find each of the following limits.
\begin{exenum}
\x
$\lim_{n\goesto\infty} \left(1+ \frac1n\right)^{2n}$,
\quad $n$ an integer.
\x
$\lim_{n\goesto\infty} \left(1+ \frac1x\right)^{-3x}$,
\quad $x$ an integer.
\end{exenum}

\ex{5.2.15}
Let $f$ be a function differentiable on some
unbounded interval $(a, \infty)$.
Prove that if
$\lim_{x\goesto\infty} [f(x) + f^\prime(x)] = L$,
then $\lim_{x\goesto\infty} f(x) = L$.
[\emph{Hint:} Consider the quotient
$\frac{e^xf(x)}{e^x}$.]

\end{exercises}
