\begin{exercises}

\ex{5.4.1}
Find the derivative with respect to $x$
of each of the following functions.
\begin{exenum}
\x
$4^{x+1}$
\x
$\log_{10}(x^2+1)$
\x
$\log_{10}4^{x+1}$
\x
$e^{x^2+x+2}$
\x
$xa^x$
\x
$2^xx^2$
\x
$xe^{-x}$
\x
$log_4 (x^2-4^x)$
\x
$x^{x-1}$
\x
$x^{(x^2)}$
\x
$(x^x)^2$.
\end{exenum}

\ex{5.4.2}
If $a$ and $b$ are positive numbers not equal
to $1$, prove that $\log_ab = \frac1{\log_ba}$.

\ex{5.4.3}
Prove that
\begin{exenum}
\x
$\ln x = (\ln a)(log_a x)$
\x
$\ln x = \frac{\log_ax}{\log_ae}$.
\end{exenum}

\ex{5.4.4}
Integrate each of the following.
\begin{exenum}
\x
$\int 7^x \; dx$
\x
$\int x^22^{3x^3+4} \; dx$
\x
$\int \frac1{x+2} \ln |x+2| \; dx$
\x
$\int \frac1x \ln \left|\frac1x\right| \; dx$
\x
$\int \log_2e^{7x-5} \; dx$
\x
$\int \frac1{x+3} 3^{\ln |x+3|} \; dx$
\x
$\int e^x5^{(e^x)} \; dx$.
\end{exenum}

\ex{5.4.5}
\begin{exenum}
\x
\xlab{5.4.5a}
Differentiate logarithmically
$y = \sqrt{\frac{(x-1)(x+3)}{(x+2)(x-4)}}$.
\x
For what values of $x$ is the differentiation
in \exref{5.4.5a} valid?
\end{exenum}

\ex{5.4.6}
If $u$ is a positive function of $x$ and $a$ is
positive but not equal to $1$,
show that
$\log_au = \frac{\ln u}{\ln a}$.

\ex{5.4.7}
Differentiate each of the following functions
with respect to $x$.
\begin{exenum}
\x
$x^{\ln x}$
\x
$(\ln x)^x$
\x
$(e^x)^{x^2+1}$
\x
$(\ln x)^{\ln x}$.
\end{exenum}

\ex{5.4.8}
\begin{exenum}
\x
Given only that $\log_a1=0$ and that
$\log_apq = \log_ap +\log_aq$,
prove that
$\log_a \left(\frac1p \right) = -\log_ap$.
\x
Then prove that
$\log_a \frac{p}{q} = \log_ap - \log_aq$.
\end{exenum}

\ex{5.4.9}
Prove all the properties listed in \thref{5.4.7}
(see Problem \exref{5.4.8}).

\ex{5.4.10}
Using the Mean Value Theorem, Theorem \thref{2.5.2},
prove that if $f^\prime(x) > 0$ for all $x$,
then $f$ is a strictly increasing function.

\ex{5.4.11}
Assume that $a  > 1$.
\begin{exenum}
\x
\xlab{5.4.11a}
Using the definition of $a^x$, show that
\[
\lim_{x\goesto\infty} a^x = \infty
.
\]
\x
\xlab{5.4.11b}
Using the result of \exref{5.4.11a}, prove that
\[
\lim_{x\goesto-\infty} a^x = 0
.
\]
\x
\xlab{5.4.11c}
Using \exref{5.4.11a}, show that
\[
\lim_{x\goesto\infty} \frac{d}{dx} a^x = \infty
.
\]
\x
\xlab{5.4.11d}
Using \exref{5.4.11b}, show that
\[
\lim_{x\goesto-\infty} \frac{d}{dx} a^x = 0
.
\]
\x
What do \exref{5.4.11a}, \exref{5.4.11b}, \exref{5.4.11c},
and \exref{5.4.11d} say geometrically about the graph
of the function $a^x$?
\end{exenum}

\ex{5.4.12}
Assume that $a_1 > a_2 > 1$.
\begin{exenum}
\x
\xlab{5.4.12a}
Using the definition of $a^x$,
show that, if $x > 0$, then ${a_1}^x > {a_2}^x$.
\x
Using \exref{5.4.12a}, show that, if $x < 0$,
then ${a_1}^x < {a_2}^x$.
\end{exenum}

\ex{5.4.13}
Evaluation of a limit of the form
$\lim_{x\goesto a} f(x)^{g(x)}$
is not obvious if any one of the following
three possibilities occurs.

(i) \quad $\lim_{x\goesto a} f(x) = \lim_{x\goesto a} g(x) = 0$.

(ii) \quad $\lim_{x\goesto a} f(x) = 1$ and
$\lim_{x\goesto a} g(x) = \infty$.

(iii) \quad $\lim_{x\goesto a} f(x) = \infty$ and
$\lim_{x\goesto a} g(x) = 0$.

These three types are usually referred to,
respectively, as the \dt{indeterminate forms}
$0^0$, $1^\infty$, and $\infty^0$.
The standard attack, akin to logarithmic differentiation,
is the following:  Let
\[
h(x) = \ln f(x)^{g(x)} = g(x) \ln f(x) =
\frac{\ln f(x)}{\frac1{g(x)}}
.
\]
One then applies L'H\^opital's Rule to the quotient,
thereby hopefully discovering that
$\lim_{x\goesto a} h(x)$ exists and what its value is.
If it does exist, it follows by the continuity
of the exponential function that
\[
e^{\left[ \lim_{x\goesto a} h(x)\right]} =
\lim_{x\goesto a} e^{h(x)}
.
\]
But, since
\[
e^{h(x)} = e^{\ln f(x)^{g(x)}} = f(x)^{g(x)}
,
\]
we therefore conclude that
\[
\lim_{x\goesto a} f(x)^{g(x)} =
e^{\left[ \lim_{x\goesto a} h(x)\right]}
,
\]
and the problem is solved.

Apply this method to evaluate the following limits.
\begin{exenum}
\x
$\lim_{x\goesto0+} x^x$
\x
$\lim_{x\goesto\infty} x^{\frac1x}$
\x
$\lim_{x\goesto0} (1+2x)^{\frac1x}$.
\end{exenum}

\end{exercises}
