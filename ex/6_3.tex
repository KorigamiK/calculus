\begin{exercises}

\ex{6.3.1}
Which of the six trigonometric functions are
odd functions and which are even functions?

\ex{6.3.2}
Derive the formulas for the derivatives of the functions
$\cot$, $\sec$, and $\csc$.

\ex{6.3.3}
Find the following derivatives.
\begin{exenum}
\x
$\ddx \sec^2 x$
\x
$\ddx \tan (2x^2 - 1)$
\x
$\ddx \ln |\sec x|$
\x
$\frac{d}{dy} \cos y \tan y$
\x
$\frac{d}{dt} (\sec^2t - 1)$
\x
$\ddx \csc (x^3 - 1)$
\x
$\ddx \tan x \cot x$
\x
$\frac{d}{dt} \ln |\cot t|$.
\end{exenum}

\ex{6.3.4}
Prove each of the following identities from the
basic identities in sine and cosine
developed in Section \secref{6.1}.
\begin{exenum}
\x
$\tan(x+y) = \frac{\tan x + \tan y}{1 - \tan x \tan y}$
\x
$\csc x = \sec \left( x - \frac{\pi}2 \right)$
\x
$\cot(a+b) = \frac{\cot a\cot b-1}{\cot a+\cot b}$
\x
$\cot(x+\pi) = \cot x$.
\end{exenum}

\ex{6.3.5}
Find the following intervals.
\begin{exenum}
\x
$\int \tan 5x \; dx$
\x
$\int \cot x \; dx$
\x
$\int e^x \sec^2 e^x \; dx$
\x
$\int \tan^2 x \; dx$ \quad
[\emph{Hint:} Use $\tan^2 x + 1 = \sec^2 x$.]
\x
$\int \tan^4 x \sec^2 x \; dx$
\x
$\int \sec^4 x \tan x \; dx$
\x
$\int \frac1{\sqrt{x}} \csc \sqrt{x} \cot \sqrt{x} \; dx$
\x
$\int \csc^4 x \; dx$.
\end{exenum}

\ex{6.3.6}
Find $\ddx (\sec x + \tan x)$ and use the result
to evaluate the integral
\[
\int \sec x \; dx = \int \sec x
\frac{\sec x + \tan x}{\sec x + \tan x} \; dx
.
\]

\ex{6.3.7}
Find $\int \tan x \; dx = \int \frac{\tan x \sec x}{\sec x} \; dx$
by substituting $u = \sec x$ and $\nxder{}ux$
in the right side.
Compare the answer obtained with Example
\exref{6.3.1b}.

\ex{6.3.8}
Draw the graph of
\begin{exenum}
\x
$\cot x$
\x
$\csc x$.
\end{exenum}

\ex{6.3.9}
Evaluate the following limits.
\begin{exenum}
\x
$\lim_{x\goesto0} \frac{\tan x}x$
\x
$\lim_{x\goesto0} x \cot 2x$.
\end{exenum}

\ex{6.3.10}
Find the tangent of the angle between
\begin{exenum}
\x
the straight lines $y-2x=1$ and $2y-x=4$.
\x
the straight lines $y+2x=2$ and $2y-x=2$.
\x
the tangent lines to the curves $y=x^2$
and $x^2+y^2=1$ at their point of
intersection in the first quadrant.
\end{exenum}

\ex{6.3.11}
What is the domain and the range of each one
of the six trigonometric functions?

\ex{6.3.12}
Evaluate each of the following indeterminate
forms (see Problem \exref{5.4.13}):
\begin{exenum}
\x
$\lim_{x\goesto0+} (\sin x)^{\tan x}$
\x
$\lim_{x\goesto0+} x^{1-\cos x}$.
\end{exenum}

\ex{6.3.13}
If $\lim_{x\goesto a} f(x) = \lim_{x\goesto a} g(x) =
\pm \infty$,
it is not immediately apparent whether or not
$\lim_{x\goesto a} (f(x)-g(x))$ exists.
Such limits are commonly called
\dt{indeterminate forms} of the type
$\infty - \infty$.
The usual method of evaluation is to express the
difference $f(x) - g(x)$ as a quotient and then to
try to find its limit.
For example, we write
\[
\frac{e^x}{e^x-1} - \frac1x =
\frac{xe^x - (e^x-1)}{x(e^x - 1)}
,
\]
and, as $x$ approaches zero, the limit of the right
side can be obtained by two applications of
L'H\^opital's Rule.  Evaluate
\begin{exenum}
\x
$\lim_{x\goesto0} \left(\frac{e^x}{e^x-1} - \frac1x\right)$
\x
$\lim_{x\goesto0} \left[\frac{(x^2+8)^{\frac13}}
{2x^2} - \frac1{x^2}\right]$
\x
$\lim_{x\goesto0} \left(\frac{x^2+3x+5}
{\sin x} - \frac5x\right)$
\x
$\lim_{t\goesto0} \left(\cot t - \frac{1-2t}t\right)$
\x
$\lim_{x\goesto0+} \left(\frac1x + \ln x\right)$
\x
$\lim_{x\goesto\frac{\pi}2} (\sec x - \tan x)$.
\end{exenum}

\end{exercises}
