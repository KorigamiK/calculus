\begin{exercises}

\ex{7.2.1}
Integrate each of the following.
\begin{exenum}
\x
$\int \cos 5x \sin 2x \; dx$
\x
$\int \cos 3x \cos x \; dx$
\x
$\int \sin 3x \sin x \; dx$
\x
$\int \cos 4z \cos 7z \; dz$
\x
$\int \cos 3x \sin \pi x \; dx$
\x
$\int \cos 2\pi y \sin \pi y \; dy$
\x
$\int \sin x \sin 6x \; dx$
\x
$\int \cos 7w \sin 17w \; dw$.
\end{exenum}

\ex{7.2.2}
\begin{exenum}
\x
\xlab{7.2.2a}
Integrate $\int \sin^3\theta \; d\theta$
by using the fact that the exponent of $\sin \theta$
is an odd positive integer.
\x
\xlab{7.2.2b}
Integrate $\int \sin^3 \theta \; d \theta$
by making use of the identity 
$\sin 3\theta = 3 \sin \theta - 4 \sin^3\theta$.
\x
Show that the answers obtained in
\exref{7.2.2a} and \exref{7.2.2b}
differ by a constant.
\end{exenum}

\ex{7.2.3}
\begin{exenum}
\x
\xlab{7.2.3a}
Integrate $\int \sin^5 2x \; dx$ by using the fact
that the exponent of the sine is an odd
positive integer.
\x
\xlab{7.2.3b}
Integrate $\int \sin^5 2x \; dx$ by using the
recursion formula given in Problem
\exref{7.1.4}.
\x
Show that the answers obtained in
\exref{7.2.3a} and \exref{7.2.3b}
differ by a constant.
\end{exenum}

\ex{7.2.4}
Integrate each of the following.
\begin{exenum}
\x
$\int \cos^3 2x \; dx$
\x
$\int \cos^4 x \; dx$
\x
$\int \sin^4 3x \; dx$
\x
$\int \sin^3 x \cos^{44} x \; dx$
\x
$\int \sin^2\theta \cos^2\theta \; d\theta$
\x
$\int \sqrt{\sin x} \cos^5 x \; dx$
\x
$\int \cos^6 3x \; dx$
\x
$\int \sin^4 y \cos^5 y \; dy$
\x
$\int \cos^2 y \sin^4 y \; dy$
\x
$\int \sin^3 x (\cos x)^{\frac52} dx$.
\end{exenum}

\ex{7.2.5}
\begin{exenum}
\x
\xlab{7.2.5a}
Integrate $\int \cos^2\theta \; d\theta$
using the identity
$\cos^2\theta = \frac12(1+\cos 2\theta)$.
\x
\xlab{7.2.5b}
Integrate $\int \cos^2\theta \; d\theta$ by parts.
\x
Show that the answers obtained in
\exref{7.2.5a} and \exref{7.2.5b}
differ by a constant.
\end{exenum}

\ex{7.2.6}
Evaluate $\int \sec^2 x \tan x \; dx$
in two different ways:
first using the fact that the secant has
an even exponent, and then using the
fact that the tangent has an odd exponent.
Show that the two solutions differ by a constant.

\ex{7.2.7}
Integrate each of the following.
\begin{exenum}
\x
$\int \tan^4 x \; dx$
\x
$\int \tan^3 4y \; dy$
\x
$\int \sec^4\theta \; d\theta$
\x
$\int \sec^3 2x \; dx$
\x
$\int \sec^4x \tan^4x \; dx$
\x
$\int \sec^3x \tan^3x \; dx$
\x
$\int \sec^4x \tan^5x \; dx$
\x
$\int \sec^6x \sqrt{\tan x} \; dx$
\x
$\int \frac{dx}{\sec x \tan x}$.
\end{exenum}

\ex{7.2.8}
\begin{exenum}
\x
\xlab{7.2.8a}
Let $n \geq 2$ be an integer, and derive
a reduction formula for
$\int \cot^nx x \; dx$
analogous to \thref{7.2.1}.
\x
Use the formula derived in \exref{7.2.8a}
to integrate $\int \cot^5 3\theta \; d\theta$.
\end{exenum}

\ex{7.2.9}
By a method analogous to that used previously
to find $\int \sec x \; dx$, prove that
\[
\int \csc x \; dx = -\ln|\csc x + \cot x| + c
.
\]

\ex{7.2.10}
\begin{exenum}
\x
Use integration by parts to derive the reduction formula
\thref{7.2.4} for $\int \csc^nx \; dx$.
\x
Use this formula to integrate
$\int \csc^6y \; dy$.
\end{exenum}

\ex{7.2.11}
Integrate each of the following.
\begin{exenum}
\x
$\int \csc^5\theta \; d\theta$
\x
$\int \sin 3x \cot 3x \; dx$
\x
$\int \cot^4y \; dy$
\x
$\int \csc^4x \cot^2x \; dx$
\x
$\int \csc^32y \cot^3 2y \; dy$
\x
$\int \csc^3\phi \cot^2\phi \; d\phi$.
\end{exenum}

\end{exercises}
