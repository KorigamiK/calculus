\begin{exercises}

\ex{8.4.1}
A solid $Q$ has a flat base which is the region
in the plane bounded by the parabola
$y^2=x$ and the line $x=4$.  Each cross-section
perpendicular to the $x$-axis is a square
with one edge lying in the base.
Find the volume of $Q$.

\ex{8.4.2}
The solid $P$ has the same base as $Q$ in
Problem \exref{8.4.1}, but each cross-section
perpendicular to the $x$-axis is a semicircular
disk with diameter lying in the base.
Compare $\mbox{\emph{vol}}(P)$.

\ex{8.4.3}
A tetrahedron is a solid with for vertices and four
flat triangular faces.  Let $T$ be a tetrahedron
which has three mutually perpendicular edges
of lengths $3$, $4$, and $10$ meeting
at a vertex.
Draw a picture of $T$ and compute its volume.

\ex{8.4.4}
The graph of the function $f(x) = \sqrt{a^2-x^2}$
is a semicircle of radius $a$.
Use this function and an integral formula
for the volume of a solid of revolution to compute
the volume of a sphere of radius $a$.

\ex{8.4.5}
Find the volume of the ellipsoid of revolution
obtained by rotating about the $x$-axis
the region bounded by the ellipse
$\frac{x^2}{a^2}+\frac{y^2}{b^2} = 1$.

\ex{8.4.6}
Sketch and find the volume of each of the solids
of revolution obtained by rotating about the
$x$-axis the region bounded by the indicated
curves and lines.
\begin{exenum}
\x
\xlab{8.4.6a}
$y=x^2-1$, $x=1$, $x=2$, and the $x$-axis.
\x
$y=\frac12x$, $x=8$, and the $x$-axis.
\x
$y=1+2x-x^2$, $x=0$, $y=0$, and $x=2$.
\x
$y=\frac1{x^2}$, $x=1$, $x=2$, and the $x$-axis.
\x
$y=1-x^2$ and the $x$-axis.
\end{exenum}

\ex{8.4.7}
Find the volume of a right circular cone of
height $h$ and with a base of radius $a$.

\ex{8.4.8}
\begin{exenum}
\x
\xlab{8.4.8a}
Find the volume of the solid of revolution obtained
by rotating about the $y$-axis the region bounded
by the $x$-axis, and the graphs of $y=x^2-1$
and $y=3$.
\x
Using \exref{8.4.8a}, find the volume generated
by rotating the region in Problem \exref{8.4.6a}
about the $y$-axis.
(Use Example \exampref{8.4.4} as a model.)
\end{exenum}

\ex{8.4.9}
Using the method of cylindrical shells,
find the volume of the solid of revolution
obtained by rotating each of the regions is
Problem \exref{8.4.6} about the $y$-axis.

\ex{8.4.10}
Sketch the region $R$ in the plane which is
bounded by the parabola $(y-1)^2 = x$,
the line $y=2$, and the $x$-axis and
$y$-axis.  Find the volume of the solid of
revolution obtained by rotating $R$ about
the $x$-axis, using
\begin{exenum}
\x
formula \thref{8.4.2} twice, i.e.,
$\pi \int_a^b y^2dx$ once with $y-1=\sqrt{x}$
and again with $y-1=-\sqrt{x}$.
\x
the counterpart of formula \thref{8.4.3}, i.e.,
the method of cylindrical shells,
for functions of $y$.
\end{exenum}

\ex{8.4.11}
Since $(x-h)^2 + y^2 = a^2$ is an equation
of the circle with radius $a$ and center
at $(h,0)$, it follows by solving for $y$
in terms of $x$ that the graph of the function
$f(x) = \sqrt{a^2-(x-h)^2}$ is a semicircle.
\begin{exenum}
\x
\xlab{8.4.11a}
Assuming that $h>a$ and using the method
of cylindrical shells, write a definite integral
for the volume of the solid torus (doughnut)
with radii $h$ and $a$.
\x
Evaluate the integral in \exref{8.4.11a} by
making the change of variable $y=x-h$
[see Theorem \thref{4.6.6}],
and using the fact that
$\int_{-a}^a \sqrt{a^2-y^2} \; dy = \frac{\pi a^2}2$
(area of a semicircle).
\end{exenum}

\ex{8.4.12}
In a solid mass of material, the infinitesimal
mass $dm$ of an infinitesimal amount of volume
$dv$ located at an arbitrary point is given by
\[
dm = \rho \; dv
,
\]
where $\rho$ is the density of the material
at that point.

Consider a cylindrical container of radius $a$
filled to a depth $h$ with a liquid whose
density is greater at the bottom and
less at the top.  Specifically, at a point a distance
$x$ below the surface the density is given by
$\rho = 2 + x$.
What is the total mass of liquid in the container?

\ex{8.4.13}
Same as Problem \exref{8.4.12},
but this time the container is a right circular cone
(apex at the bottom) of height $h$ and base
of radius $a$ which is filled to the top.

\end{exercises}
