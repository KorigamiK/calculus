\begin{exercises}

\ex{9.8.1}
For each of the values of $n$ indicated,
compute the Taylor polynomial $T_n$
which approximates the function $f$ near
the number $a$.
\begin{exenum}
\x
$f(x) = \frac1{x+1}$, $a=0$, $n=0$, $1$, and $2$.
\x
$f(x) = \frac1{1+x^2}$, $a=0$, $n=0$, $2$, and $4$.
\x
$f(x) = \frac1{1+x^2}$, $a=1$, $n=0$, $1$, and $2$.
\x
$f(x) = \sqrt{x+1}$, $a=3$, $n=1$, $2$, and $3$.
\x
$f(x) = \sin x$, $a=\frac{\pi}4$, $n=0$, $1$, and $2$.
\end{exenum}

\ex{9.8.2}
Compute the formula,
for an arbitrary nonnegative integer $n$,
for the approximating Taylor polynomial
to the function $f$ about the number $a$.
\begin{exenum}
\x
$f(x) = \cos x$, \quad $a=0$
\x
$f(x) = \ln x$, \quad $a=1$.
\end{exenum}

\ex{9.8.3}
For $n=0$, $1$, and $2$, compute the Taylor
polynomial $T_n$ which approximates the
function $f$ near $0$.
Draw the graphs of the three polynomials together
with the graph of $f$.
\begin{exenum}
\x
$f(x) = e^x$
\x
$f(x) = \cos x$.
\end{exenum}

\ex{9.8.4}
Let $p$ be a polynomial in $x$ of degree
$\leq m$; i.e., the function $p$ is defined
by an equation
\[
p(x) = a_0 + a_1x + \cdots + a_mx^m
,
\]
and let $T_n$ be the Taylor polynomial which
approximates $p$ near an arbitrary real number
$a$.  Prove, as a simple consequence of
Taylor's formula with the remainder,
that $p(x) = T_n(x)$, for every real number $x$
provided $n \geq m$.

\ex{9.8.5}
For each of the values of $n$ indicated,
compute the approximation $T_n$ to the
polynomial $p$ near the number $a$.
\begin{exenum}
\x
$p(x) = x^2+3x-1$, $a=2$, $n=1$, $2$, and $3$.
\x
$p(x) = 2x^3-5x^2+3$, $a=0$, $n=1$, $2$, and $3$.
\x
$p(x) = x^4+3x^2+x+2$, $a=0$,
$n=3$, $4$, and $17$.
\x
$p(x) = x^3-1$, $a=1$, $n=2$, $3$, and $4$.
\end{exenum}

\ex{9.8.6}
Prove that, for every real number $x$,
\[
\cos x = \sum_{i=0}^\infty (-1)^i
\frac{x^{2i}}{(2i)!} =
1 - \frac{x^2}2 + \frac{x^4}{4!} - \frac{x^6}{6!} + \cdots
.
\]

\ex{9.8.7}
For each of the following functions,
compute the Taylor series about $a$.
\begin{exenum}
\x
$e^x$, if $a=0$
\x
$\frac{e^x}{e^2}$, if $a=2$
\x
$\arctan x$, if $a=0$.
\end{exenum}

\ex{9.8.8}
\begin{exenum}
\x
Compute the cubic Taylor polynomial $p(x)$
which approximates the function $\frac1{x+2}$
for values of $x$ near the number $1$.
\x
Show that, for every $x$ in the interval
$[0,2]$, the approximation $p(x)$ differs
in absolute value from $\frac1{x+2}$
by less that $0.04$.
\end{exenum}

\ex{9.8.9}
Show that $\sin x$ differs in absolute value
from the approximation $x - \frac{x^3}6$
by no more that $\frac{\pi^5}{15\cdot2^8}
= 0.025$ (approximately) for every $x$ in the interval
$\left[ -\frac{\pi}2, \frac{\pi}2 \right]$.

\ex{9.8.10}
The Taylor approximation $T_n$ to a function $f$
about the number $a$ is frequently called the
\emph{best polynomial approximation
of degree $\leq n$} to the function $f$ near $a$
because it can be shown that $T_n$ is the
only polynomial of degree $\leq n$ with the property
that, as $x$ approaches $a$, the difference
$f(x) - T_n(x)$ approaches zero
faster than $(x-a)^n$.

Prove the following part of the above assertion:
If $f$ has continuous $(n+1)$st derivative
in an open interval containing $a$, then
$\lim_{x\goesto{a}} \frac{f(x)-T_n(x)}
{(x-a)^n} = 0$.

\ex{9.8.11}
What cubic polynomial best approximates
$x^4-2x^3+3x-3$ near $x=2$?

\ex{9.8.12}
Another statement of Taylor's Theorem which
gives a different form for the remainder
is the following:
\emph{Let $f$ be a function with continuous
$(n+1)\mathrm{st}$ derivative at every point
of the interval $[a,b]$.  Then
\[
f(b) = f(a) + f^\prime(a)(b-a) + \cdots +
\frac1{n!} f^{(n)} (a)(b-a)^n
\]
\[
+ \int_a^b \frac{(b-t)^n}{n!} f^{(n+1)}
(t) \; dt
.
\]}
\begin{exenum}
\x
\xlab{9.8.12a}
Using integration by parts, show that
\[
\int_a^b \frac{(b-t)^n}{n!} f^{(n+1)} (t) \; dt
\]
\[
= - \frac1{n!} f^{(n)} (a)(b-a)^n +
\int_a^b \frac{(b-t)^{n-1}}{(n-1)!}
f^{(n)} (t) \; dt
.
\]
\x
Using induction on $n$ and the result of part
\exref{9.8.12a}, prove the above form of
Taylor's Theorem in which the remainder
appears as an integral.
\end{exenum}

\ex{9.8.13}
Let $f$ have a continuous second derivative
at every point of an interval containing the
number $a$ in its interior, and let
$f^\prime(a) = 0$.
Show that $f$ has a local maximum value
at $a$ if $f^{\prime\prime} (a) < 0$, and a
local minimum value at $a$ if $f^{\prime\prime}
(a) > 0$.
[\emph{Hint:} Use the Taylor Formula
$f(x) = T_1(x) + R_1$ and the fact that,
if a continuous function is positive (or negative)
at $a$, then it is positive (or negative) near $a$.]

\end{exercises}
